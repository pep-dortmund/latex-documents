% vim: spelllang=de
\documentclass[
  paper=a4,
  fontsize=12pt,
  parskip=half,
  headinclude=true,
]{scrartcl}


\usepackage[margin=2cm, bottom=3cm]{geometry}

\usepackage{fontspec}
\setsansfont{Source Sans Pro}
\usepackage{microtype}
\renewcommand{\familydefault}{\sfdefault}
\usepackage[main=ngerman]{babel}
\usepackage[locale=DE]{siunitx}



\usepackage[autostyle]{csquotes}

\usepackage{enumitem}
\setlist{noitemsep, topsep=0.0pt}
\usepackage{booktabs}
\usepackage{xcolor}
\usepackage{calc}
\usepackage[colorlinks=true,urlcolor=blue!50!black]{hyperref}

\usepackage{siunitx}
\usepackage{enumerate}

\renewcommand*\sectionformat{TOP~\thesection{}~}

\title{Protokoll der Mitgliederversammlung des Vereins PeP at al. e.V.}
\author{Protokollfürerïn}
\date{%
  Tag, Datum\\
  \begin{tabular}{l l}
    Beginn: & xx:xx\\
    Ende: & xx:xx\\
  \end{tabular}
}

\begin{document}

\maketitle

\subsection*{Anwesenheitsliste}

Es nehmen insgesamt XX Teilnehmerïnnen teil:

% Alphabetisch nach Nachnahmen
John Doe, Jane Doe 


\section{Begrüßung}

XX eröffnet um XX:XX die Mitgliederversammlung (MV) und stellt die fristgerechte Einladung fest.
Das Protokoll wird von YY geführt.

\section{Beschluss der Tagesordnung}

Die Tagesordnung wird einstimmig angenommen.

\section{Berichte}

\subsection{Finanzen}
\subsection{Stipendien}
\subsection{Sommer- und Alumniakademie}
\subsection{Toolbox Workshop}
\subsection{Workshops}
\subsection{Absolventenfeier}
\subsection{PhysiKon}
\subsection{Stammtische}
\subsection{Going Abroad}
\subsection{Sonstiges}

\section{Bericht der Kassenprüfer}

\section{Entlastung des Vorstands}

\section{Wahlen des Vorstands und der Kassenprüfer}


\subsection{Wahl des geschäftsführenden Vorstands}

Es werden folgende Personen für die jeweiligen Ämter vorgeschlagen:

\begin{tabular}{l l}
  1. Vorsitzender: & \\
  2. Vorsitzender: & \\
  Finanzreferentin: & \\
\end{tabular}

Die drei Ämter werden einstimmig, en bloc gewählt.
Die Wahl wird angenommen.

\subsection{Wahl der Vorstandsposten (VP) Koordination, Mitgliederverwaltung und Ressourcenverwaltung}

Es werden vorgeschlagen:

\begin{tabular}{l l}
  Koordination: & \\
  Mitgliederverwaltung: & \\
  Ressourcenverwaltung: & \\
\end{tabular}

Die drei Ämter werden einstimmig, en bloc gewählt, die Wahl wird angenommen.


\subsection{Wahl der VPs Begabtenförderung, Fortbildung, Toolbox-Workshop}

Es werden vorgeschlagen:

\begin{tabular}{l l}
  Begabtenförderung (2 Posten): & \\
  Fortbildung: & \\
  Toolbox-Workshop: & \\
\end{tabular}

Die vier Ämter werden einstimmig, en bloc gewählt, die Wahl wird angenommen.


\subsection{Wahl der VPs Öffentlichkeitsarbeit, Alumniarbeit, Sommerakademie}

Es werden vorgeschlagen:

\begin{tabular}{l l}
  Öffentlichkeitsarbeit: & \\
  Alumniarbeit: (2 Posten) & \\
  Sommerakademie: & \\
\end{tabular}

Die vier Kandidaten und Kandidatinnen werden en bloc einstimmig, die Wahl wird angenommen.

\subsection{Wahl der VPs Wirtschaft und Lehramt}

Es werden vorgeschlagen:

\begin{tabular}{l l}
  Wirtschaft: & \\
  Lehramt:  & \\
\end{tabular}

Die beiden Kandidaten werden en bloc einstimmig gewählt, die Wahl wird angenommen.


\subsection{Wahl der Kassenprüfer}
Es werden vorgeschlagen:

\begin{itemize}
  \item XX
  \item XX
\end{itemize}

\section{Haushalt}


\begin{center}
\begin{tabular}{l r r}
  \toprule
  Absolventenfeier       & XX\,€ & jährlich bis auf Widerruf\\
  Deutschlandstipendien: & XX\,€ & einmalig\\
  Going Abroad 2021      & XX\,€ & einmalig\\
  Initiativstipendien:   & XX\,€ & jährlich bis auf Widerruf\\
  PhysiKon               & XX\,€ & jährlich bis auf Widerruf\\
  Sommerakademie         & XX\,€ & jährlich bis auf Widerruf\\
  Workshops              & XX\,€ & jährlich bis auf Widerruf\\
  \midrule
  \bfseries Gesamt & XX\,€ & \\
  \bottomrule
\end{tabular}
\end{center}

\section{Abschluss}

XY schließt die MV um 20:05.

\end{document}



































