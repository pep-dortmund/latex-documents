\documentclass{pepnewsletter}

\usepackage[ngerman]{babel}

\usepackage{microtype}
\usepackage{enumitem}
\usepackage{caption}

\usepackage{wrapfig}
\usepackage{graphicx}

\usepackage[autostyle]{csquotes}

\hypersetup{
  unicode,
  pdftitle={PeP et al. Newsletter},
  pdfcreator={},
  pdfproducer={},
}
\usepackage{bookmark}

\title{Newsletter}
\date{Mai 2015}

\begin{document}

\section*{Liebe Studierende, liebe Alumni,}
wie ihr an diesem ersten Newsletter erkennen könnt, ist unser Verein einigen Veränderungen unterworfen. Wir haben uns in den letzten zehn Jahren von einer anfänglichen Idee zu einer festen Instanz der Fakultät Physik entwickelt und damit eine überaus positive Entwicklung erfahren.

Daneben haben sich auf der diesjährigen Mitgliederversammlung ein paar personelle Veränderungen in der Geschäftsführung unseres Vereins ergeben.

Der alte Vorstand, bestehend aus Christophe Cauet, Julian Wishahi und Tobias Brambach war der Ansicht, dass der Weg für eine neue Generation der Geschäftsführung freigemacht werden musste und stand daher nicht mehr für eine weitere Amtsperiode zur Verfügung.

An dieser Stelle möchte sich der neue Vorstand im Namen des Vereins herzlich bei Christophe, Julian und Tobias für ihr langjähriges Engagement in der Geschäftsführung des Vereins bedanken. Dank eures Einsatzes gibt es nun einen gut funktionierenden Vorstand, eine regelmäßige Sommerakademie, sowie einen Stammtisch für Alumni und Bildungsangebote für Studierende.

Wer aber ist jetzt die neue Geschäftsführung von PeP et al.?

Die neu gewählte Führung des Vereins besteht aus Henning Moldenhauer, Alex Birnkraut und Vanessa Müller.
Henning übernimmt das Amt von Christophe und ist nun erster Vorsitzender. Alex hat Julians Position, das Amt des zweiten Vorsitzenden, inne und Vanessa löst Tobias ab. Sie ist die neue Finanzreferentin.

Bei Fragen meldet euch einfach bei uns oder kommt vorbei.

Herzliche Grüße\\[1ex]
Henning Moldenhauer\\
Alex Birnkraut\\
Vanessa Müller\\
im Namen aller aktiven Mitglieder

\section*{Die neue Geschäftsführung}

\begin{wrapfigure}{r}{.21\textwidth}
	\vspace{-20pt}
	\begin{center}
		\includegraphics[width=.21\textwidth]{example-image-a}
	\end{center}
	\vspace{-20pt}
	\caption*{Henning Moldenhauer}
	\vspace{-10pt}
\end{wrapfigure}

\textit{\enquote{Ich kenne PeP seit Beginn meines Studiums und habe die Begeisterung erlebt, mit der der scheidende Vorstand den Verein gestaltet hat.\\
Mich motiviert die kompetente und zielgerichtete Zusammenarbeit im gesamten Vorstand und ich freue mich auf die neuen Aufgaben in der Geschäftsführung des Vereins.}}

\begin{wrapfigure}{l}{.21\textwidth}
	\vspace{-20pt}
	\begin{center}
		\includegraphics[width=.21\textwidth]{example-image-a}
	\end{center}
	\vspace{-20pt}
	\caption*{Alex Birnkraut}
	\vspace{-10pt}
\end{wrapfigure}

\textit{\enquote{Ich habe PeP auf meiner ersten Fahrt zur Sommerakademie im fünften Semester richtig kennengelernt und war sofort fasziniert von dem Engagement, das alle Organisatoren zeigten. Ich hoffe die Arbeit des alten Vorstandes in dieser Hinsicht fortführen zu können und weitere spannende Projekte zu initiieren.}}

\begin{wrapfigure}{r}{.21\textwidth}
	\vspace{-20pt}
	\begin{center}
		\includegraphics[width=.21\textwidth]{example-image-a}
	\end{center}
	\vspace{-20pt}
	\caption*{Vanessa Müller}
	\vspace{-10pt}
\end{wrapfigure}

\textit{\enquote{Ich habe PeP durch die Sommerakademie kennen gelernt und war sofort beeindruckt durch den Zusammenhalt im Vorstand, wie auch im Verein selbst. Außerdem gefallen mir die Projekte von PeP und auch, dass man wirklich etwas bewegen kann. Ich hoffe, in Zukunft die hervorragende Arbeit des ehemaligen Vorstands mit weiter führen zu können.}}

\end{document}
