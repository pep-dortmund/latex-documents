\documentclass[parskip=half, fontsize=10pt, paper=a5]{scrartcl}

\usepackage[margin=0.5cm, left=3cm]{geometry}

\usepackage{fontspec}
\setmainfont{Akkurat Office}

\usepackage[ngerman]{babel}

\usepackage{csquotes}
\usepackage{fontawesome5}

\usepackage{graphicx}
\usepackage{enumitem}

\pagestyle{empty}

\usepackage{tikz}

\usepackage{xcolor}
\definecolor{tu}{RGB}{132,182,24}
\definecolor{physik}{RGB}{8,81,162}


\usepackage[colorlinks,urlcolor=tu]{hyperref}



\begin{document}
\raggedright

\begin{tikzpicture}[remember picture, overlay, shift=(current page.north west)]
  \fill[tu] (0.5cm, 0) rectangle (2.5cm, -\paperheight);
  \node[anchor=north west, inner sep=0.1cm] at (0.5cm, -0.5cm) {%
    \includegraphics[width=1.8cm]{schwingung_positiv_white.pdf}
  };
\end{tikzpicture}

\begin{center}
\textbf{\Large Einladung zur Absolventenfeier\\ der Jahrgänge 2019 – 2022}
\end{center}
\vspace{0.5cm}

Liebe Absolventin oder lieber Absolvent,

zu deinem in den Jahren 2019 – 2022 an der TU Dortmund erworbenen Abschluss möchten
wir dir ganz herzlich gratulieren.
Die Fakultät Physik und der Alumni-Verein PeP et al.\ e.\,V.\ sind der Meinung, dass dies ein Grund zum Feiern ist.
Daher möchten wir dich und deine Angehörigen zur diesjährigen Absolventenfeier einladen.

\vspace{0.25cm}
\textcolor{tu}{\textbf{\large Informationen}}
\begin{description}[style=multiline, leftmargin=5em]
  \item[Wann] 1. April 2023, 15:00 Uhr
  \item[Wo] Campus Nord\\
    \textbf{Audimax}\\
    Vogelpothsweg 87, 44227 Dortmund
\end{description}

\vspace{0.25cm}
\textcolor{tu}{\textbf{\large Anmeldung}}

Wenn du an der Absolventenfeier teilnehmen möchtest, melde dich dazu bitte unter
\begin{center}
  \large
  \href{https://registration.pep-dortmund.org/events/absol23/registration}{registration.pep-dortmund.org} 
\end{center}
an.

Wir bitten um dein Verständnis, dass auf Grund der großen Anzahl an erwarteten Teilnehmenden vorerst nur zwei weitere
Gäste mitgebracht werden können.

Die Feier beginnt um 16:00 Uhr. Vorher möchten wir gemeinsam mit allen Absolvent*innen ein Gruppenfoto aufnehmen.
Wir bitten daher sich \emph{spätestens} um 15:30 Uhr im Foyer des Gebäudes einzufinden.
Ab 15 Uhr stehen für euch Kaffee und Kekse bereit.


\vspace*{\fill}
\begin{center}
\includegraphics[height=1cm]{tu.pdf}%
\hfill%
\includegraphics[height=1cm]{pep.pdf}%
\end{center}

\newpage
\begin{tikzpicture}[remember picture, overlay, shift=(current page.north west)]
  \fill[tu] (0.5cm, 0) rectangle (2.5cm, -\paperheight);
  \node[anchor=north west, inner sep=0.1cm] at (0.5cm, -0.5cm) {%
    \includegraphics[width=1.8cm]{schwingung_positiv_white.pdf}
  };
\end{tikzpicture}
\begin{center}
\textbf{\Large Einladung zur Absolventenfeier\\ der Jahrgänge 2019 – 2022}
\end{center}

\vspace*{\fill}
\textcolor{tu}{\textbf{\large Vorläufiges Programm}}
\begin{description}[leftmargin=0.5em]
  \item[15:00 Uhr] Empfang
  \item[15:30 Uhr] Gruppenfoto aller Absolvent*innen
  \item[16:00 Uhr] Begrüßung \\[0.5\baselineskip]
    {\small 
    Kevin Schmidt (Vorsitzender PeP et al.~e.\,V.)\\
    Prof.~Dr. Markus Betz   (Studiendekan der Fakultät Physik)\\
    Grußworte des Rektorats
    }
  \item[16:30 Uhr] Ehrung der Diplom- und Master-Absolvent*innen
  \item[17:00 Uhr] Vortrag \enquote{Was kommt nach dem Studium?}\\[0.5\baselineskip]
    {\small Dr. Julian Wishahi (Head of Data Intelligence \& Analytics, Deichmann)}
  \item[17:20 Uhr] Ehrung der Promotions-Absolvent*innen
  \item[17:40 Uhr] Preisverleihung der Wilhelm und Else Heraeus-Stiftung
  \item[17:50 Uhr] Vortrag der Preisträgerin
  \item[18:00 Uhr] Verabschiedung
\end{description}

\begin{center}
  Im Anschluss an die Veranstaltung laden wir dich und deine Begleitung herzlich zu Canapés und Sekt ein!
\end{center}


\vspace*{\fill}

\begin{center}
\includegraphics[height=1cm]{tu.pdf}%
\hfill%
\includegraphics[height=1cm]{pep.pdf}%
\end{center}
  
\end{document}
