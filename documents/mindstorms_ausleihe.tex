\documentclass[parskip=half]{scrartcl}
\pagestyle{empty}

\usepackage[margin=2.5cm]{geometry}
\usepackage{xcolor}
\usepackage{hyperref}
\usepackage{fontspec}
\usepackage{polyglossia}
\setdefaultlanguage{german}

\usepackage{fontawesome}

\hypersetup{
  pdfauthor={Kai Brügge}
}

\begin{document}
% \title{Ausleihformular für Lego Minstorms}
\section*{Ausleihformular für Lego Minstorms}

Zum Ausleihen eines Lego Mindstorms Set sind folgende Informationen vollständig auszufüllen.
\begin{Form}
\subsection*{Ausgeliehen an:}
\renewcommand{\LayoutTextField}[2]{% label, field
  #2\\\vspace{-4pt}{\scriptsize#1}%
}
\TextField[name=NAME, width=\textwidth]{Name, Vorname}\\
\\
\TextField[name=street, width=\textwidth]{Straße, Hausnummer}\\
\\
\TextField[name=city, width=\textwidth,]{PLZ, Stadt}

\subsection*{Ausgeliehen von:}
\TextField[name=pepname, width=\textwidth]{Name, Vorname}

\subsection*{Lego Set:}
Ausgeliehen wird der das Set mit der Nummer \TextField[name=setnum, width=1cm,borderstyle=U]{}\\
Zum Ausgabedatum ist das Set komplett: \CheckBox[name=ausgabe]{}
\\
\\
\renewcommand{\LayoutTextField}[2]{#2}
Ausgabedatum: \TextField[name=datumausgabe, width=3cm,borderstyle=_]{TT.MM.JJJJ}\hspace{1cm} Spätestes Rückgabedatum: \TextField[name=datumspeatesterueckgabe, width=3cm,borderstyle=U]{TT.MM.JJJJ}\\
\\
\renewcommand{\LayoutTextField}[2]{% label, field
  #2\\\vspace{-4pt}{\scriptsize#1}%
}
\TextField[name=sig, width=\textwidth]{Datum, Unterschrift}
\\
\\
\noindent\makebox[\linewidth]{\rule{\paperwidth}{0.4pt}}
\\
\renewcommand{\LayoutTextField}[2]{#2}
\subsection*{Rückgabe:}
Zurückgegeben am : \TextField[name=datereuckgabe, width=3cm,borderstyle=_]{TT.MM.JJJJ}\hspace{1cm}\\
Zum Rückgabedatum ist das Set komplett: \CheckBox[name=reuckgabe]{}
\\
\\
\renewcommand{\LayoutTextField}[2]{% label, field
  #2\\\vspace{-4pt}{\scriptsize#1}%
}
\TextField[name=signature, width=\textwidth]{Datum, Unterschrift}\\
\\
\TextField[name=signaturepep, width=\textwidth]{Datum, Unterschrift PEP Mitglied}

\end{Form}

\end{document}
