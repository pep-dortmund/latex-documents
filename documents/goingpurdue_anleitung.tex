\documentclass[
  paper=a4,
  fontsize=12pt,
  DIV=16,
  headheight=52pt,
  footheight=45pt,
  headinclude,
  parskip=full,
]{scrartcl}

\usepackage{fontspec}
\setsansfont{Source Sans Pro}
\renewcommand{\familydefault}{\sfdefault}

\usepackage{polyglossia}
\setmainlanguage{german}

\usepackage{csquotes}

\usepackage{microtype}
\usepackage{graphicx}

\usepackage{scrlayer-scrpage}
\pagestyle{scrheadings}
\setkomafont{pagehead}{\normalfont}
\setkomafont{pagefoot}{\normalfont\footnotesize}
\ohead{\includegraphics[height=1.5cm]{pep.pdf}}
\ihead{%
  \large\bfseries PeP et al.\ e.\,V.\\%
  Begabtenförderungsprogramm –\\
  Going Abroad - Purdue%
}
\cfoot{}
\ifoot{%
  PeP et al.~e.\,V.\\
  Begabtenförderung\\
  \url{www.pep-dortmund.org}
}


\usepackage{titling}
\usepackage{booktabs}

\usepackage{xcolor}
\usepackage{calc}
\usepackage[colorlinks=true,urlcolor=blue!50!black]{hyperref}

\date{31. Juli 2014}

\begin{document}
\textbf{\Huge\sffamily Bachelorarbeit in Purdue, Indiana}\\[0.5\baselineskip]
\textbf{\Large\sffamily Eine Anleitung was alles im Vorfeld tun ist}

Du hast die Zusage zur Teilnahme am Going Abroad Programm für Purdue erhalten.
Mit der alleinigen Zusage ist es leider nicht getan.
Es gibt noch einige wichige Aufgaben, die du vorher erledigen musst.
Im folgenden werden dir die wichtigsten Fragen beantwortet.
Solltest du nicht ausreichend informiert sein, oder noch weitere Fragen haben,
kannst du dich an Henning Moldenhauer (CP-O1-190) persönlich oder per Mail
(henning.moldenhauer@pep-dortmund.org) wenden.

\subsubsection*{Wie bekomme ich das richtige Visum für die USA?}
Das für dich richtige Visum ist das J1 Visum.
Der Prozess ist etwas komplexer und wird dir im folgenden dargelegt.

Zuerst musst du die \textbf{Einverständniserklärung der Fakultät} von deinem
betreuenden Professor \textbf{unterschreiben lassen} und bei Henning im Büro
abgeben.
Diese Erklärung ist sehr wichtig und wird schnellstmöglich von uns benötigt.
Erst mit ihr können wir den Visumsprozess an der Purdue University starten, sowie
deine Unterkunft vor Ort organisieren.

Ebenfalls entscheidend für die Beantragung deines Visums ist der
\textbf{Arbeitsvertrag}, den du mit der Einverständniserklärung unterschrieben
bei uns abgeben musst.
% Noch etwas besser begründen.

Sobald wir die Dokumente unterschrieben zurück bekommen haben, werden diese an
Andreas Jung in die USA weitergeleitet.
% Visumsprozess noch einmal genau durchgehen und hier rein schreiben

Mit den unterschriebenen Dokumenten musst du uns auch eine
\textbf{Kopie des Reisepasses} einreichen, damit auf amerikamischer Seite die
Dokumente für das Visum mit den richtigen Daten ausgefüllt werden können.

Während dieser Vorbereitungen wirst du ein paar Mails erhalten, die du
ebenfalls zügig beantworten solltest.
% was genau für Mails? von wem? was wird in etwa gefragt?

\subsubsection*{Warum muss ich mich Krankenversichern?}
Du musst dich selbst zwingend Krankenversichern.

\subsubsection*{Wie komme ich nach West Lafayette, Indiana?}

\subsubsection*{Wo schlafe ich?}

\end{document}
