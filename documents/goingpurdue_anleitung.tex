\documentclass[
  paper=a4,
  fontsize=12pt,
  DIV=16,
  headheight=52pt,
  footheight=45pt,
  headinclude,
  parskip=full,
]{scrartcl}

\usepackage{fontspec}
\setsansfont{Source Sans Pro}
\renewcommand{\familydefault}{\sfdefault}

\usepackage{polyglossia}
\setmainlanguage{german}

\usepackage{csquotes}

\usepackage{microtype}
\usepackage{graphicx}

\usepackage{scrlayer-scrpage}
\pagestyle{scrheadings}
\setkomafont{pagehead}{\normalfont}
\setkomafont{pagefoot}{\normalfont\footnotesize}
\ohead{\includegraphics[height=1.5cm]{pep.pdf}}
\ihead{%
  \large\bfseries PeP et al.\ e.\,V.\\%
  Begabtenförderungsprogramm –\\
  Going Abroad - Purdue%
}
\cfoot{}
\ifoot{%
  PeP et al.~e.\,V.\\
  Begabtenförderung\\
  \url{www.pep-dortmund.org}
}

\usepackage[hyphens]{url}

\usepackage{titling}
\usepackage{booktabs}

\usepackage{xcolor}
\usepackage{calc}
\usepackage[colorlinks=true,urlcolor=blue!50!black]{hyperref}

\date{10. October 2017}

\begin{document}
\textbf{\Huge\sffamily Bachelorarbeit in Purdue, Indiana}\\[0.5\baselineskip]
\textbf{\Large\sffamily Eine Anleitung was alles im Vorfeld tun ist}

Du hast die Zusage zur Teilnahme am Going Abroad Programm für Purdue erhalten.
Mit der alleinigen Zusage ist es leider nicht getan.
Es gibt noch einige wichige Aufgaben, die du vorher erledigen musst.
Im folgenden werden dir die wichtigsten Fragen beantwortet.
Solltest du nicht ausreichend informiert sein, oder noch weitere Fragen haben,
kannst du dich an Henning Moldenhauer (CP-O1-190) persönlich oder per Mail
(henning.moldenhauer@pep-dortmund.org) wenden.

\subsubsection*{Wie bekomme ich das richtige Visum für die USA?}
Das für dich richtige Visum ist das J1 Visum.
Der Prozess ist etwas komplexer und wird dir im folgenden dargelegt.

Zuerst musst du die \textbf{Einverständniserklärung der Fakultät} von deinem
betreuenden Professor \textbf{unterschreiben lassen} und bei Henning im Büro
abgeben.
Diese Erklärung ist sehr wichtig und wird schnellstmöglich von uns benötigt.
Erst mit ihr können wir den Visumsprozess an der Purdue University starten, sowie
deine Unterkunft vor Ort organisieren.

Ebenfalls entscheidend für die Beantragung deines Visums ist der
\textbf{Arbeitsvertrag}, den du mit der Einverständniserklärung unterschrieben
bei uns abgeben musst.
% Noch etwas besser begründen.

Sobald wir die Dokumente unterschrieben zurück bekommen haben, werden diese an
Andreas Jung in die USA weitergeleitet.
Er wird dann an der Purdue Universität dafür sorgen, dass du dort auch angenommen
wirst. Dazu wird er bei verschiedenen Stellen, wie dem Business Office,
dem Visa and Immigration Office und dem Office of International Students and
Scholars (ISS), alles notwendige in die Wege leiten.
% werden hier noch weitere Dokumente benötigt?

Mit den unterschriebenen Dokumenten musst du uns auch eine
\textbf{Kopie des Reisepasses} einreichen, damit auf amerikamischer Seite die
Dokumente für das Visum mit den richtigen Daten ausgefüllt werden können.

Während dieser Vorbereitungen wirst du ein paar Mails erhalten, die du
ebenfalls zügig beantworten solltest.
Unter anderem wirst du dazu aufgefordert werden die
\textbf{MyISS J Request E-forms} auszufüllen.
Hier wirst du zu deinen bisherigen Reisen in die USA und in andere Länder
gefragt, musst Daten zu deinen letzten Arbeitgebern und zum Studium, sowie
deine Passnummer angeben.
Es wird auch nach deinen Reisedaten gefragt werden. Hier reicht es erst einmal
Mai 2018 bis August 2018 anzugeben.

Wenn du hier noch weitere Fragen hast, kannst du beim US Department of State
noch weitere Informationen erhalten:
\url{https://travel.state.gov/content/visas/en/forms/ds-160--online-nonimmigrant-visa-application/frequently-asked-questions.html}

\subsubsection*{Warum muss ich mich Krankenversichern?}
Eine Sommerschule in den USA ist nur mit einem J-1 Visum möglich.
Der Visaprozess erfordert zwingend das Vorhandensein einer
\textbf{US tauglichen Krankenversicherung}.
In den Staaten reicht deine Versicherung nicht aus und du musst selbst eine
Zusatzversicherung abschließen.
Hier gibt es verschiedene Möglichkeiten:
\begin{itemize}
  \item Der DAAD bietet über die Continentale (sitzt in Dortmund) solche
  Zusatzversicherungen für etwa 160 € im Monat an.
  \item Es kann auch eine Krankenversicherung in den USA zugekauft werden, hier
  gibt es zum Beispiel die folgenden Varianten:
  \begin{itemize}
    \item \url{www.compassstudenthealthinsurance.com/compare_international_insurance_plans.php}
    Eine US Krankenversicherung ist immer mit sehr viel Auswahl verbunden.
    Wir empfehlen eine most-inclusive insurance, um bei tatsächlicher Krankheit
    den hohen Kosten des US Gesundheitssystems vorzubeugen.
    Compass Benchmark Plus in der entsprechenden Altersklasse liegt - mit
    Selbstbeteiligung - bei etwa 200 € pro Monat.
    \item Das ISS von Purdue empfiehlt diese zwei Optionen:
    \begin{itemize}
      \item \url{www.compassstudenthealthinsurance.com/}
      \item \url{www.isoa.org/}
    \end{itemize}
  \end{itemize}
\end{itemize}

\subsubsection*{Okay, aber muss ich das alles selber zahlen?}
Nein, dazu hast du dich ja bei uns auf das Austauschprogramm beworben.
Du wirst einen Arbeitsvertrag bekommen der den Großteil deiner Kosten vor Ort
decken sollte.
Über das Geld was du erhältst kannst du frei verfügen.
Allerdings musst du davon die Krankenversicherung zahlen.

Das Geld für Flug und Visum, musst zu vorstrecken.
Sobald du jedoch eine Rechnung darfür hast, kannst du dir das Geld bei PeP
wieder holen.
Du solltest einen möglichst günstigen Flug zu bekommen, mehr als 1500 € können
nicht erstattet werden.
Die Kosten für das Visum werden voll übernommen.
% An und Abreise zur Botschaft auch?

Die Kosten für deine Unterkunft an der Purdue University werden ebenfalls
vollständig übernommen.

\subsubsection*{Wie komme ich nach West Lafayette, Indiana?}
Mit dem Flugzeug geht es nach Indianapolis, Indiana. Und von da aus weiter
mit dem Bus nach West Lafayette.
Eine empfohlene Busverbindung erhälst du nachdem wir die Flugdaten abgeklärt
haben.
Den Flughafen für Hin- und Ruckflug kannst du dir aussuchen.
Hast du ein passendes Angebot gefunden, so buchst du eigenständig nach Absprache
mit uns. Das Geld für den Flug und den Bus wird dir nach erhalt der Rechnung
überwiesen.
%Ab wann darf man anreisen, wann Abreisen?


\subsubsection*{Wo schlafe ich?}
Während deines Aufenthaltes an der Purdue University wirst du in einem
Doppelzimmer mit einem gleichgeschlechtlichen Studenten in der Hawkins Hall
untergebracht.
Hier kannst du deine Unterkunft schon einmal besichtigen und erste Fragen
beantwortet bekommen:
\url{https://www.housing.purdue.edu/Housing/Residences/Hawkins/index.html}
%Hat man einen Safe, also wie sieht es mit Wertsachen aus?

\subsubsection*{Wie werde ich versorgt?}
%Muss man sich selbst um alles kümmern?
%Gibt es einen Kühlschrank im Zimmer?
%Was habe ich für Möglichkeiten?

\subsubsection*{Wie läuft das mit der Bachelorarbeit?}
%Wer korrigiert die Bachelorarbeit?
%Wo wird die die Bachelorarbeit abgegeben?
%Auf welcher Sprache muss die Bachelorarbeit geschrieben werden?
%Wann beginnt die Bachelorarbeit?
%Ist die Benotung der Bachelorarbeit anders als wenn ich sie an der Tu schreibe?
%Wann wird der Bachelorvortrag gehalten und was muss er enthalten?

\subsubsection*{Wie sieht der Arbeitsalltag aus?}
%Hier würde ich grob beschreiben, wie so ein überlicher Tag abläuft wird und
%wie es mit Freizeit aussieht.
%Damit sich die Studenten darauf einstellen können.

\end{document}
