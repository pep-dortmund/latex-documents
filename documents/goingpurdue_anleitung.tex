\documentclass[
  paper=a4,
  fontsize=12pt,
  DIV=16,
  headheight=52pt,
  footheight=45pt,
  headinclude,
  parskip=full,
]{scrartcl}

\usepackage{fontspec}
\setsansfont{Source Sans Pro}
\renewcommand{\familydefault}{\sfdefault}

\usepackage{polyglossia}
\setmainlanguage{german}

\usepackage{csquotes}

\usepackage{microtype}
\usepackage{graphicx}

\usepackage{scrlayer-scrpage}
\pagestyle{scrheadings}
\setkomafont{pagehead}{\normalfont}
\setkomafont{pagefoot}{\normalfont\footnotesize}
\ohead{\includegraphics[height=1.5cm]{pep.pdf}}
\ihead{%
  \large\bfseries PeP et al.\ e.\,V.\\%
  Begabtenförderungsprogramm –\\
  Going Abroad - Purdue%
}
\cfoot{}
\ifoot{%
  PeP et al.~e.\,V.\\
  Begabtenförderung\\
  \url{www.pep-dortmund.org}
}

\usepackage[hyphens]{url}

\usepackage{titling}
\usepackage{booktabs}

\usepackage{xcolor}
\usepackage{calc}
\usepackage[colorlinks=true,urlcolor=blue!50!black]{hyperref}

\date{10. October 2017}

\begin{document}
\textbf{\Huge\sffamily Bachelorarbeit in Purdue, Indiana}\\[0.5\baselineskip]
\textbf{\Large\sffamily Eine Anleitung was alles im Vorfeld tun ist}

Du hast die Zusage zur Teilnahme am Going Abroad Programm für Purdue erhalten.
Mit der alleinigen Zusage ist es leider nicht getan.
Es gibt noch einige wichige Aufgaben, die du vorher erledigen musst.
Im folgenden werden dir die wichtigsten Fragen beantwortet.
Solltest du nicht ausreichend informiert sein, oder noch weitere Fragen haben,
kannst du dich an Henning Moldenhauer (CP-O1-190) persönlich oder per Mail
(henning.moldenhauer@pep-dortmund.org) wenden.

\subsubsection*{Wie bekomme ich das richtige Visum für die USA?}
Das für dich richtige Visum ist das J1 Visum.
Der Prozess ist etwas komplexer und wird dir im folgenden dargelegt.

Zuerst musst du die \textbf{Einverständniserklärung der Fakultät} von deinem
betreuenden Professor \textbf{unterschreiben lassen} und bei Henning im Büro
abgeben.
Diese Erklärung ist sehr wichtig und wird schnellstmöglich von uns benötigt.
Erst mit ihr können wir den Visumsprozess an der Purdue University starten, sowie
deine Unterkunft vor Ort organisieren.

Ebenfalls entscheidend für die Beantragung deines Visums ist der
\textbf{Arbeitsvertrag}, den du mit der Einverständniserklärung unterschrieben
bei uns abgeben musst.

Das \textbf{Visiting Scholar Information Form} musst du mit den angefragten Daten
füllen und ebenfalls bei uns abgeben.

Sobald wir die Dokumente unterschrieben zurück bekommen haben, werden diese an
Andreas Jung in die USA weitergeleitet.
Er wird dann an der Purdue Universität dafür sorgen, dass du dort auch angenommen
wirst. Dazu wird er bei verschiedenen Stellen, wie dem Business Office,
dem Visa and Immigration Office und dem Office of International Students and
Scholars (ISS), alles notwendige in die Wege leiten.

Mit den unterschriebenen Dokumenten musst du uns auch eine
\textbf{Kopie des Reisepasses} einreichen, damit auf amerikamischer Seite die
Dokumente für das Visum mit den richtigen Daten ausgefüllt werden können.
Falls du deinen Reisepass noch beantragen musst, brauchen wir deinen
\textbf{vollständigen Namen}, wie er im Pass steht, sowie \textbf{dein
Geburtsdatum}. Den Reisepass solltest du dann idealerweise im Eilverfahren
beantragen.

Während dieser Vorbereitungen wirst du ein paar Mails erhalten, die du
ebenfalls zügig beantworten solltest.
Unter anderem wirst du dazu aufgefordert werden die
\textbf{MyISS J Request E-forms} auszufüllen.
Hier wirst du zu deinen bisherigen Reisen in die USA und in andere Länder
gefragt, musst Daten zu deinen letzten Arbeitgebern und zum Studium, sowie
deine Passnummer angeben.
Es wird auch nach deinen Reisedaten gefragt werden. Hier reicht es erst einmal
Mai 2018 bis August 2018 anzugeben.

Wenn du hier noch weitere Fragen hast, kannst du beim US Department of State
noch weitere Informationen erhalten:
\url{https://travel.state.gov/content/visas/en/forms/ds-160--online-nonimmigrant-visa-application/frequently-asked-questions.html}

\subsubsection*{Warum muss ich mich Krankenversichern?}
Eine Sommerschule in den USA ist nur mit einem J-1 Visum möglich.
Der Visaprozess erfordert zwingend das Vorhandensein einer
\textbf{US tauglichen Krankenversicherung}.
In den Staaten reicht deine Versicherung nicht aus und du musst selbst eine
Zusatzversicherung abschließen.
Hier gibt es verschiedene Möglichkeiten:
\begin{itemize}
  \item Der DAAD bietet über die Continentale (sitzt in Dortmund) solche
  Zusatzversicherungen für etwa 160 € im Monat an.
  \item Es kann auch eine Krankenversicherung in den USA zugekauft werden, hier
  gibt es zum Beispiel die folgenden Varianten:
  \begin{itemize}
    \item \url{www.compassstudenthealthinsurance.com/compare_international_insurance_plans.php}
    Eine US Krankenversicherung ist immer mit sehr viel Auswahl verbunden.
    Wir empfehlen eine most-inclusive insurance, um bei tatsächlicher Krankheit
    den hohen Kosten des US Gesundheitssystems vorzubeugen.
    Compass Benchmark Plus in der entsprechenden Altersklasse liegt - mit
    Selbstbeteiligung - bei etwa 200 € pro Monat.
    \item Das ISS von Purdue empfiehlt diese zwei Optionen:
    \begin{itemize}
      \item \url{www.compassstudenthealthinsurance.com/}
      \item \url{www.isoa.org/}
    \end{itemize}
  \end{itemize}
\end{itemize}

\subsubsection*{Okay, aber muss ich das alles selber zahlen?}
Nein, dazu hast du dich ja bei uns auf das Austauschprogramm beworben.
Du wirst einen Arbeitsvertrag bekommen der den Großteil deiner Kosten vor Ort
decken sollte.
Über das Geld was du erhältst kannst du frei verfügen.
Allerdings musst du davon die Krankenversicherung zahlen.

Das Geld für Flug und Visum, musst zu vorstrecken.
Sobald du jedoch eine Rechnung darfür hast, kannst du dir das Geld bei PeP
wieder holen.
Du solltest einen möglichst günstigen Flug zu bekommen, mehr als 1500 € können
nicht erstattet werden.
Die Kosten für das Visum werden voll übernommen.
% An und Abreise zur Botschaft auch?

Die Kosten für deine Unterkunft an der Purdue University werden ebenfalls
vollständig übernommen.

\subsubsection*{Wie komme ich nach West Lafayette, Indiana?}
Mit dem Flugzeug geht es nach Indianapolis, Indiana. Und von da aus weiter
mit dem Bus nach West Lafayette.
Eine empfohlene Busverbindung erhälst du nachdem wir die Flugdaten abgeklärt
haben.
Den Flughafen für Hin- und Ruckflug kannst du dir aussuchen.
Hast du ein passendes Angebot gefunden, so buchst du eigenständig nach Absprache
mit uns. Das Geld für den Flug und den Bus wird dir nach erhalt der Rechnung
überwiesen.

Der Aufenthalt in Unterkünften der Uni ist vom 15. Mai bis zum 15. August
möglich. Das ist genau der Zeitraum des Summer Break. Wenn du vor oder nach dem
Programm noch in den USA auf eigene Kosten reisen willst, wende dich noch einmal
an uns, da lässt sich sicher eine individuelle Regelung finden.


\subsubsection*{Wo schlafe ich?}
Während deines Aufenthaltes an der Purdue University wirst du in einem
Doppelzimmer mit einem gleichgeschlechtlichen Studenten in der Hawkins Hall
untergebracht.
Hier kannst du deine Unterkunft schon einmal besichtigen und erste Fragen
beantwortet bekommen:
\url{https://www.housing.purdue.edu/Housing/Residences/Hawkins/index.html}
%Hat man einen Safe, also wie sieht es mit Wertsachen aus?

\subsubsection*{Wie werde ich versorgt?}
In deiner Unterkunft gibt es eine große Gemeinschaftsküche.
Dort kannst du dich mit anderen Studenten arrangieren und ihr könnt gemeinsam
kochen.

Am Campus gibt es verschiedene Dining Courts, Markets, Restaurants und Cafes.
Hier wirst du sicher deinen Favoriten finden.
Einen ersten Überblick kannst du dir hier verschaffen
\url{https://dining.purdue.edu/AboutUs/map.html}.

Beliebte Orte findest du hier \url{https://lafayettedaily.com/purdue-food/}.

\subsubsection*{Wie läuft das mit der Bachelorarbeit?}
Du wirst deine Bachelorarbeit während deines Aufenthaltes an der Purdue
University unter der Aufsicht und Betreuung von Andreas Jung schreiben.
Hier kannst du dir aussuchen ob du die Arbeit auf Deutsch oder auf Englisch
verfassen willst.

Auch auf deutscher Seite hast du einen Professor, der dich betreuen wird.
Da er allerdings nicht vor Ort ist, kannst du ihm nur per Mail Fragen stellen.
Deine Arbeit musst du hinterher bei deinem Betreuer in Deutschland abgeben.
Schließlich bist du ja an der TU Dortmund eingeschrieben und machst auch dort
deinen Abschluss.

Wenn du Fragen hast, die du weder Andreas Jung, noch deinem betreuenden Professor
stellen willst, kannst du dich vertrauensvoll an PeP wenden.
Wir werden dir dann einen Ansprechpartner aus unseren Reihen zu Seite stellen,
der dir bei deinen Fragen weiterhelfen wird.

Deinen Bachelorvortrag musst du anschließend an deinen Auslandsaufenthalt in
Deutschland am Lehrstuhl deines deutschen Professors halten.
Du solltest dort - wie bei jedem anderen Bachelorvortrag auch - dein Thema
darstellen, sowie die Ergebnisse deiner Arbeit präsentieren.
Neben dem fachlichen Teil ist es auch erforderlich, dass du zehn bis fünfzehn
Minuten über deine Erfahrungen an der Purdue University berichtest.

Die Benotung deiner Arbeit folgt den gleichen Regeln, als wenn du die Arbeit an
der TU geschrieben hättest.
Der einzige Unterschied ist, dass Andreas Jung deinem betreuenden Professor
sehr wahrscheinlich eine Empfehlung für die Note aussprechen wird.

\subsubsection*{Wie sieht der Arbeitsalltag aus?}
Um sich den Arbeitsalltag besser vorstellen zu können, sind hier zwei Berichte
von Studenten, die in Andreas Jung's Labor gearbeitet haben:

\paragraph{On simulations of thermal performance of the upgraded pixel detector}
My primary responsibilities include producing thermal simulations of anything
the group needed, such as a replication of our carbon fiber calibration
experiment or a more current design of our CO2 cooling structure.
Whenever Professor Jung begins an experiment involving heated detectors and
fibers, he often relies on me to create (mostly from scratch) thermal simulations for
comparison.
For the past year and a half, designs of CMS inner detectors by physicists at
the LHC have been sent to us for simulation.
We perform the necessary tests and finite element analyses of their designs
with some of Professor Jung’s suggestions and report our findings to the
experts at Fermilab and CERN.

\paragraph{On work to built prototypes of pixel detector supports}
As an undergraduate researcher at Purdue Silicon Detector Labs I do a large
variety of work, from programming to machining.
On a day to day basis my work generally consists of the measurement of carbon
fiber laminate dimensions on a coordinate measuring machine.
Measurement include the thickness, done with a touch probe, and bow, done with
an optical probe.
The process of measuring these values was originally largely manual and very
time consuming.
One of my tasks is to make the process more efficient and automated.
I also am the labs gantry operator.
This means that I operate and program the robotic gantry used for a variety of
precision applications.
I also participate in the production of the carbon fiber sheets from cutting
to layout to curing.

Im folgenden erhältst du eine Liste mit typischen Aufgaben, die für dich
geeignet sein könnten.
So kannst du dir schon mal ein Bild davon machen, was dich erwartet und vielleicht
findest du auch schon ein Thema welches dich interessiert.
\begin{itemize}
  \item Visual Inspections of HDI/Modules (mostly completed since phase I
  upgrade is completed)
  \item IV tests of modules
  \item Gluing/encapsulation on the gantry
  \item Coding for gantry and other tooling (Labview)
  \item Coding for data analysis (ROOT, C++)
  \item Using CMM to measure parts/features (sensors and CF)
  \item Measure out/Lay up/ Cure CF plys
  \item Test thermal conductivity of Carbon foam and fiber
  \item Heat transfer of thermal wing (experiment)
  \item Finite Element Analysis on many structures (ANSYS)
  \item Design new tooling in CAD programs (Inventor, Solidworks, ANSYS)
  \item Machine new tools in the machine shop
  \item Soldering of electrical components
  \item Setting up new machinery/equipment
  \item Learn how to do experiments and work in a clean room
  \item Work with other Purdue research centers, e.g. CMSC and Birck
  \item Collaborate with Cornell/ other worldwide universities and institutions
  \item Learn how to work in a real lab environment and get experience without
  tremendous pressures
\end{itemize}

\end{document}
