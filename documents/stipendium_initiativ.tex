\documentclass[
  paper=a4,
  fontsize=12pt,
  DIV=16,
  headheight=52pt,
  footheight=45pt,
  headinclude,
  parskip=full,
]{scrartcl}

\usepackage{fontspec}
\setsansfont{Source Sans Pro}
\renewcommand{\familydefault}{\sfdefault}

\usepackage{polyglossia}
\setmainlanguage{german}

\usepackage{microtype}
\usepackage{graphicx}

\usepackage{scrlayer-scrpage}
\pagestyle{scrheadings}
\setkomafont{pagehead}{\normalfont}
\setkomafont{pagefoot}{\normalfont\footnotesize}
\ohead{\includegraphics[height=1.5cm]{pep.pdf}}
\ihead{%
  \large\bfseries PeP et al.\ e.\,V.\\%
  Begabtenförderungsprogramm –\\
  Initiativstipendien%
}
\cfoot{}
\ifoot{%
  PeP et al.~e.\,V.\\
  Begabtenförderung\\
  \url{www.pep-dortmund.org}
}
\ofoot{%
	Christian Arauner P2-01-423\\
	Rene-Marcel Lehner P2-01-423\\
	Anno Knierim P2-01-423%
}


\usepackage{titling}
\usepackage{booktabs}

\usepackage{xcolor}
\usepackage{calc}
\usepackage[colorlinks=true,urlcolor=blue!50!black]{hyperref}

\newcommand\MyTextField[2][]{\TextField[#1, backgroundcolor=black!10, charsize=0pt, borderwidth=0]{#2}}
\renewcommand*{\LayoutTextField}[2]{\makebox[\widthof{#1: }][l]{#1: }%
\raisebox{0.8\baselineskip}{\raisebox{-\height}{#2}}}


\begin{document}
\section*{Bewerbung}

Bitte reiche Deine Bewerbung bis zum Quartalsende (31. März, 30. Juni,
30. September oder 31. Dezember) beim PeP-Vorstand
(Christian Arauner, Rene-Marcel Lehner oder Anno Knierim) ein,
addressiert an:

PeP et al. e.V.\\
Christian Arauner\\
P2-01-423\\[0.5\baselineskip]
Technische Universität Dortmund\\
Otto-Hahn-Str.~4\\
44221 Dortmund

Um für das Bewerbungsverfahren zugelassen zu werden, muss Dein Antrag bestehen aus:
\begin{itemize}
  \item einem vollständigen, tabellarischen \textbf{Lebenslauf},
  \item einem 1- bis 2-seitigen \textbf{Motivationsschreiben}, in welchem Du
	  begründest warum Dein Projekt förderungswürdig ist, wer dich vor Ort
	  betreut und in welchem Zeitraum es stattfinden soll,
  \item eine \textbf{Selbstdarstellung}, aus welcher dein soziales/
	  gesellschaftliches/fakultätsinternes und/oder sonstiges Engagement hervorgeht.
  \item einer \textbf{Aufschlüsselung} Deiner Kosten, sodass wir abschätzen können,
	  ob eine Förderung notwendig ist.
  \item einer kurzen \textbf{Projektbeschreibung}, die auf etwa einer halben Seite
	  darstellt, was genau deine Aufgabe sein wird.
\end{itemize}

\textbf{%
  Eine Bewerbung ist nur möglich, wenn du an der Technischen Universität Dortmund
  als Student eingeschrieben bist und Physik im Studiengang
  \emph{Bachelor of Science Physik},
  \emph{Bachelor of Science Lehramt Physik (BfP)},
  \emph{Bachelor of Science Medizinphysik},
  \emph{Master of Science Physik},
  \emph{Master of Science Lehramt Physik (MfP)} oder
  \emph{Master of Science Medizinphysik} studierst.
}

Bitte reiche nur DIN A4 Dokumente ein. Klammere oder hefte diese nicht und nutze auch
keine Klarsichtfolien. Unvollständige Bewerbungen werden nicht berücksichtigt.

Für weitere Fragen stehen Dir Thorben Menne und
Max Linhoff (CP-03-156) zur Verfügung.
Du erreichst sie per Mail an begabtenfoerderung@pep-dortmund.org.
\end{document}
