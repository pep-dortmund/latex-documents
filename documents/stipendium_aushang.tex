\documentclass[
  paper=a4,
  fontsize=12pt,
  DIV=16,
  headheight=52pt,
  footheight=45pt,
  headinclude,
  parskip=full,
]{scrartcl}

\usepackage{fontspec}
\setsansfont{Source Sans Pro}
\renewcommand{\familydefault}{\sfdefault}

\usepackage{polyglossia}
\setmainlanguage{german}

\usepackage{csquotes}

\usepackage{microtype}
\usepackage{graphicx}

\usepackage{scrlayer-scrpage}
\pagestyle{scrheadings}
\setkomafont{pagehead}{\normalfont}
\setkomafont{pagefoot}{\normalfont\footnotesize}
\ohead{\includegraphics[height=1.5cm]{pep.pdf}}
\ihead{%
  \large\bfseries PeP et al.\ e.\,V.\\%
  Begabtenförderungsprogramm –\\
  Deutschlandstipendien%
}
\cfoot{}
\ifoot{%
  PeP et al.~e.\,V.\\
  Begabtenförderung\\
  \url{www.pep-dortmund.org}
}


\usepackage{titling}
\usepackage{booktabs}

\usepackage{xcolor}
\usepackage{calc}
\usepackage[colorlinks=true,urlcolor=blue!50!black]{hyperref}

\date{31. Juli 2014}

\begin{document}
\textbf{\Huge\sffamily Stipendium für Physikstudierende}\\[0.5\baselineskip]
\textbf{\Large\sffamily Studienförderung für  Physikerinnen in Dortmund}

{\large Der Verein der Physikstudierenden und ehemaligen Physikstudierenden \enquote{PEP et al. e.\,V.}
vergibt auch im Wintersemester 2015/2016 wieder Stipendien an Dortmunder Physikerinnen und Physiker.
Dieses von der Bundesregierung unterstützte Programm wird zur
Hälfte aus Vereinsmitteln und zur anderen Hälfte aus dem Bundeshaushalt finanziert.
}

Das Programm richtet sich an Studierende mit überdurchschnittlichen Studienleistungen und dient explizit der Förderung des wissenschaftlichen Nachwuchses der Dortmunder Fakultät.

Die finanzielle Förderung beträgt 300 € monatlich und ist auf ein Jahr befristet.
Die Vergabe der Stipendien erfolgt leistungsorientiert und einkommensunabhängig.
Es werden fünf Stipendien vergeben.

Eine Bewerbung ist möglich, wenn Du an der Technischen Universität Dortmund als Studierender eingeschrieben bist und Physik im Studiengang Bachelor of Science, Bachelor
Lehramt Physik (BfP) oder Medizinphysik studierst und zum Wintersemester im dritten
oder fünften Fachsemester bist.

Eine Bewerbung ist zwischen dem 15. August und dem 15. September 2015 möglich.
Alle nötigen Informationen und Dokumente erhältst du auf unserer Internetseite

\begin{center}
  \Huge\bfseries\url{https://stipendium.pep-dortmund.org}
\end{center}

Für persönliche Fragen stehen dir Alex Birnkraut und Vanessa Müller (P1-O1-310) zur
Verfügung.
\end{document}
