\documentclass[
  fontsize=12pt,
  paper=a4,
  DIV14,
  parskip,
]{scrartcl}

\usepackage{fontspec}
\setmainfont{Fira Sans}
\renewcommand\familydefault\sfdefault

\usepackage{polyglossia}
\setmainlanguage{german}

\usepackage[autostyle]{csquotes}

\usepackage{graphicx}

\renewcommand*\thesection{\S{} \arabic{section}}


\begin{document}

\textbf{\huge Die Vereinssatzung}

\section{Name, Sitz und Geschäftsjahr}

\begin{enumerate}
  \item Der Verein führt den Namen \enquote{Physikstudierende und ehemalige
    Physikstudierende der Technischen Universität Dortmund et al.}, im
    folgenden \enquote{PeP et al.}$\,$genannt. Er ist beim Amtsgericht
		Dortmund eingetragen. Mit der Eintragung erhält der Verein den Zusatz
    \enquote{e.\,V.}.
	\item Der Verein hat seinen Sitz in Dortmund.
	\item Das Geschäftsjahr ist das des Kalenderjahres. Für das erste Jahr wird
		ein Rumpfwirtschaftsjahr gebildet.
\end{enumerate}

\section{Zweck des Vereins}

\begin{enumerate}
	\item Zweck des Vereins ist die Unterstützung der Fakultät Physik der
		Technischen Universität Dortmund in Forschung und Lehre, sowie die
		Verbreitung von Erkenntnissen der physikalischen Forschung.
	\item Der Satzungszweck wird insbesondere verwirklicht durch die Förderung
		des Kontakts und Ehrfahrungsaustauschs zwischen der Fakultät und ihren
		Absolventen, der Mitglieder untereinander und mit allen interessierten
		gesellschaftlichen Gruppen. Basis hierfür bilden das jährliche Treffen,
		ein Mitgliederverzeichnis, sowie Durchführung oder Förderung
		wissenschaftlicher Veranstaltungen.
\end{enumerate}

\section{Gemeinnützigkeit}

\begin{enumerate}
	\item Der Verein \enquote{PeP et al.} politisch unabhängig und ohne
		Erwerbsinteressen und verfolgt ausschließlich und unmittelbar gemeinnützige
		Zwecke im Sinne der §§52ff der Abgabenordnung 1977.
	\item Der Verein ist selbstlos tätig; er verfolgt nicht in erster Linie
		eigenwirtschaftliche Zwecke.
	\item Mittel des Vereins dürfen nur für satzungsgemäße Zwecke verwendet
		werden. Mitglieder erhalten keine Zuwendungen aus Mitteln des Vereins.
\end{enumerate}

\section{Erwerb der Mitgliedschaft}

\begin{enumerate}
	\item Der Verein \enquote{PeP et al.} hat ordentliche und
		außerordentliche Mitglieder.
	\item Ordentliche Mitglieder können alle aktiven und ehemaligen Studierenden,
		Absolventen, Lehrende, und Angestellte der Fakultät Physik der Technischen
		Universität Dortmund werden.
	\item Außerordentliche Mitglieder können alle an der Fakultät Physik
		interessierte natürliche und juristische Personen werden.
	\item Die ordentliche und außerordentliche Mitgliedschaft wird durch eine
		formlose schriftliche oder elektronische Beitrittserklärung beantragt.
		Über die Aufnahme entscheidet der Vorstand. Ein Aufnahmeanspruch besteht
		jedoch nicht. Der Vorstand ist berechtigt, die Aufnahme in den Verein ohne
		Angabe von Gründen abzulehnen.
\end{enumerate}

\section{Rechte und Pflichten der Mitglieder}

\begin{enumerate}
	\item Die Mitglieder sind verpflichtet, die Ziele und Interessen des
		Vereins zu unterstützen.
	\item Die Mitglieder sind berechtigt, die Einrichtungen des Vereins zu
		benutzen und an den Veranstaltungen teilzunehmen.
\end{enumerate}

\section{Mitgliedsbeiträge}

\begin{enumerate}
	\item Es ist ein Jahresbeitrag zu entrichten. Die Höhe des Beitrags und
		die Zahlungsweise regelt eine von der Mitgliederversammlung beschlossene
		Beitragsordnung.
	\item In begründeten Einzelfällen kann der Vorstand Beiträge stunden oder
		ganz oder teilweise erlassen. Näheres regelt die nach §6 (2) beschlossene
		Beitragsordnung.
	\item Eine Aufnahmegebühr wird nicht erhoben.
\end{enumerate}

\section{Beendigung der Mitgliedschaft}

\begin{enumerate}
	\item Die ordentliche und außerordentliche Mitgliedschaft im Verein endet
		durch freiwilligen Austritt, Ausschluss oder Tod jeweils zum Ende eines
		Kalenderjahres.
	\item Der Austritt aus dem Verein erfolgt durch schriftliche Mitteilung an
		den Vorstand.
	\item Bei Vorliegen schwerwiegender Gründe kann ein Mitglied durch die
		Mitgliederversammlung mit sofortiger Wirkung ausgeschlossen werden,
		wenn z.B. eine Schädigung des Ansehens des Vereins vorliegt.
		Vor der Beschlussfassung zum Ausschluss durch den Vorstand ist dem
		Mitglied Gelegenheit zu geben, sich innerhalb einer angemessenen Frist
		zu rechtfertigen. Der Ausschließungsbeschluss ist dem Mitglied schriftlich
		zuzustellen. Innerhalb von 14 Tagen nach Zustellung kann das Mitglied
		hiergegen Einspruch erheben.
		Über den Einspruch entscheidet der Vorstand. Erst nach der Entscheidung
		des Vorstandes kann das Mitglied die Rechtsmäßigkeit des Ausschlusses im
		ordentlichen Rechtsweg überprüfen lassen.
\end{enumerate}

\section{Organe des Vereins}

\begin{enumerate}
	\item Organe des Vereins sind:
	\begin{enumerate}
		\item die Mitgliederversammlung,
		\item der Vorstand,
		\item die Projektbeauftragten.
	\end{enumerate}
\end{enumerate}

\section{Mitgliederversammlung}

\begin{enumerate}
	\item Oberstes Organ ist die Mitgliederversammlung. Soweit nicht in dieser
		Satzung ausdrücklich andere Zuständigkeiten geregelt sind, ist die
		Mitgliederversammlung für alle Angelegenheiten des Vereins zuständig.
	\item Alle ordentlichen Mitglieder des Vereins \enquote{PeP et al.} sind
		in der Mitgliederversammlung antrags- und stimmberechtigt. Die
		Mitgliederversammlung beschließt über:
	\begin{enumerate}
		\item Entgegennahme der Berichte des Vorstandes und des Kassenprüfers,
		\item Entlastung des Vorstandes,
		\item Wahl des Vorstandes,
		\item Wahl der Projektbeauftragten,
		\item Wahl des Kassenprüfers,
		\item Änderung der Satzung,
		\item Auflösung des Vereins,
		\item Ausschluss von Mitgliedern.
	\end{enumerate}
	\item Der Vorstand lädt mindestens einmal im Jahr zur Mitgliederversammlung
		unter Angabe von Ort und Zeitpunkt schriftlich oder in elektronischer Form
		mindestens einen Monat vorher ein. Ein Mitglied des Vorstandes organisiert
		das Treffen. Die Leitung obliegt dem Vorstand oder einem gewählten
		Vertreter aus der Mitte der Mitgliederversammlung.
	\item Die Mitgliederversammlung ist unabhängig von der Zahl der teilnehmenden
		Mitglieder beschlussfähig, wenn gemäß §9 (3) ordentlich eingeladen wurde.
		Ihre Beschlüsse werden mit Stimmenmehrheit gefasst.
	\item Über die Mitgliederversammlung und die in ihr getroffenen Beschlüsse
		ist ein schriftliches Protokoll auszufertigen, welches durch den
		Versammlungsleiter und den Protokollführer zu unterzeichnen ist. Dieses
		Protokoll ist allen Mitgliedern des Vereins \enquote{PeP et al.} in
		schriftlicher oder elektronischer Form zugänglich zu machen.
	\item Bei Abstimmung und Wahlen entscheidet die einfache Stimmenmehrheit
		der abgegebenen gültigen Stimmen der anwesenden ordentlichen Mitglieder.
		Im Falle der Stimmengleichheit bei Abstimmungen entscheidet der Vorsitzende,
		bei Wahlen das Los.
	\item Beschlüsse der Mitgliederversammlung über eine vorzeitige Abberufung
		des Vorstandes, über Änderungen der Satzung sowie die Auflösung des Vereins
		bedürfen einer Mehrheit von 3/4 der anwesenden ordentlichen Mitglieder.
	\item Die Mitgliederversammlung kann eine Geschäftsordnung beschließen. Sie
		beschließt darüber hinaus über die grundsätzlichen Richtlinien in der
		Arbeit und Aufgaben des Vereins.
	\item Die Mitgliederversammlung beschließt eine Beitragsordnung, gemäß §6 (2).
	\item Eine Mitgliederversammlung kann von 2/5 aller außerordentlichen
		Mitglieder einberufen werden und muss dem Vorstand angezeigt werden.
\end{enumerate}

\section{Vorstand}

\begin{enumerate}
	\item Der Vorstand im Sinne des §26 BGB besteht aus dem Vorsitzenden, dem
		stellvertretenden Vorsitzenden und dem Finanzreferenten.\\
		Die Vertretungsmacht des Vorstandes ist mit Wirkung gegen Dritte in der
		Weise beschränkt, dass zum Erwerb oder Verkauf, zur Belastung und zu allen
		sonstigen Verfügungen über Grundstücke oder grundstücksgleichen Rechten,
		sowie zur Aufnahme eines Kredites von mehr als 1.000,00 Euro die Zustimmung
		der Mitgliederversammlung erforderlich ist.
	\item Der Verein wird durch zwei Vorstandsmitglieder rechtsgeschäftlich
		vertreten.
	\item Der Vorstand wird von der Mitgliederversammlung auf ein Jahr gewählt.
		Wiederwahl ist möglich. Er bleibt nach Ablauf seiner Amtszeit solange im
		Amt, bis ein neuer Vorstand gewählt worden ist. Wird zwischen zwei
		Mitgliederversammlungen eine Wahl erforderlich, so kann sie schriftlich
		erfolgen. Als Wahlperiode gilt dann die Zeit bis zur nächsten
		Mitgliederversammlung.
	\item Zur konkreten Durchführung von Veranstaltungen im Namen von \enquote{PeP
		et al.} kann die Mitgliederversammlung Positionen für
		Projektbeauftragte im Vorstand einrichten. Die Projektbeauftragten haben
		Stimmrecht im Vorstand.
	\item In Sitzungen, zu denen der Vorstand oder einer seiner Stellvertreter
		schriftlich unter Angabe der Tagesordnung mindestens 5 Werktage vorher
		einberuft, ist zur Beschlussfähigkeit die Anwesenheit der Mehrheit der
		Vorstandsmitglieder nötig. Beschlüsse werden mit Mehrheit der anwesenden
		Vorstandsmitglieder gefasst.
\end{enumerate}

\section{Projektbeauftragte}

\begin{enumerate}
	\item Die Projektbeauftragten werden von der Mitgliederversammlung gewählt.
	\item Bleiben Posten für Projektbeauftragte vakant, kann die
		Mitgliederversammlung den Vorstand damit beauftragen einen geeigneten
		Kandidaten zu finden und bis zur nächsten Mitgliederversammlung den Posten
		kommissarisch zu besetzen.
 	\item Bei umfangreichen Projekten kann die Mitgliederversammlung zusätzliche
		Vertreter wählen.
	\item Die Projektbeauftragten oder ihre Vertreter haben Stimmrecht bei
		Vorstandssitzungen.
	\item Die Projektbeauftragten können durch die Mitgliederversammlung mit
		finanziellen Vollmachten im Rahmen der für das jeweilige Projekt nötigen
		Aufwendungen ausgestattet werden.
		\begin{enumerate}
			\item In diesem Falle hat der Projektbeauftragte Rechenschaft über die
				getätigten Transaktionen abzulegen.
		\end{enumerate}
\end{enumerate}

\section{Satzungsänderung und Auflösung des Vereins}

\begin{enumerate}
	\item Satzungsänderungen und Auflösung können von der Mitgliederversammlung
		nur beschlossen werden, wenn bei Einberufung der Mitgliederversammlung
		gemäß §9 (3) und §9 (10) hierauf besonders hingewiesen wurde und wenn eine
		3/4 Mehrheit aller abgegebenen Stimmen dafür eintritt.
	\item Im Falle der Auflösung oder Aufhebung des Vereins \enquote{PeP et
		al.} oder bei Wegfall seines steuerbegünstigten Zweckes fällt
		etwaiges Vereinsvermögen an die Gesellschaft der Freunde der Universität
		Dortmund e.V. (Sitz: Märkische Str. 120, 44141 Dortmund) unter der Auflage
		das etwaige Vereinsvermögen von \enquote{PeP et al.} ausschließlich und
		unmittelbar für steuerbegünstigte Zwecke zu verwenden.
\end{enumerate}

\section{Schlussbestimmung}

\begin{enumerate}
	\item Die Satzung wurde von der Mitgliederversammlung des Vereins am
		15.12.2007 in Dortmund beschlossen und tritt mit Eintragung in das
		Handelsregister Dortmund in Kraft.
\end{enumerate}

\end{document}
