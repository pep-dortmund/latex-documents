\documentclass[
  fontsize=12pt,
  paper=a4,
  DIV14,
  parskip,
]{scrartcl}

\usepackage{fontspec}
\setmainfont{Source Sans Pro}
\renewcommand\familydefault\sfdefault

\usepackage{polyglossia}
\setmainlanguage{german}

\usepackage[autostyle]{csquotes}

\usepackage{todonotes}
\usepackage{comment}

\usepackage{enumitem}
\setlist[enumerate, 1]{label=(\arabic*), labelindent=2em, itemindent=!, leftmargin=!}
\setlist[enumerate, 2]{label=(\alph*), parsep=0.5ex}
\setlist[itemize]{parsep=0.5ex}

\usepackage{graphicx}

\usepackage[
  colorlinks=true,
  urlcolor=blue
]{hyperref}

\renewcommand*\thesection{\S{} \arabic{section}}

% Verwendete Quellen:
% - https://www.baden-wuerttemberg.datenschutz.de/wp-content/uploads/2018/05/Praxisratgeber-f%C3%BCr-Vereine.pdf
% - Datenschutzerklärung der DPG: https://www.dpg-physik.de/datenschutz/

\begin{document}

\textbf{\huge Datenschutzerklärung}

Dem Verein Physikstudierende und ehemalige Physikstudierende der Technischen
Universität Dortmund et al. e.V. (nachfolgend PeP et al.) ist der sorgfältige
und sichere Umgang mit den persönlichen Daten von Vereinsmitgliedern und
Dritten unter Beachtung gesetzlicher Datenschutzvorgaben (DSGVO) sehr
wichtig.
Mit dieser Erklärung möchte PeP et al. Sie über Art, Umfang und Zweck der
Verarbeitung Ihrer personenbezogenen Daten informieren.
Mit Ihrer Zustimmung zur Datenschutzerklärung von PeP et al. geben Sie Ihr
Einverständnis, dass der Verein Ihre personenbezogenen Daten zur Erfüllung
der Vereinszwecke bzw. zur Durchführung von Vereinsaktivitäten erheben,
verarbeiten und nutzen darf.
Diese Einwilligung können Sie jederzeit mit Wirkung für die Zukunft
widerrufen.

PeP et al. verfolgt keine wirtschaftlichen Interessen. Der Verein hält den
Kontakt und fördert den Erfahrungsaustausch zwischen der Fakultät und ihren
Absolventen, der Mitglieder untereinander und mit allen interessierten
gesellschaftlichen Gruppen.
Basis hierfür bilden regelmäßige Treffen, ein Mitgliederverzeichnis, sowie
Durchführung und Förderung wissenschaftlicher Veranstaltungen.

Der Sitz von PeP et al. ist in Dortmund. Im § 2 der Satzung findet sich der
ausformulierte Zweck des Vereins, dem PeP et al. verpflichtet ist.

Aus Gründen der besseren Lesbarkeit wird in diesem Text die männliche Form
verwendet; selbstverständlich ist stets sowohl die weibliche als auch die
männliche Form gemeint.

\section*{Verantwortliche Stelle}

Der Verantwortliche im Sinne der Datenschutz-Grundverordnung (DSGVO) ist\\
PeP et al. e.V.\\
Otto-Hahn-Straße\\
44227 Dortmund

\url{https://pep-dortmund.org}\\
\url{https://registration.pep-dortmund.org/}\\
kontakt@pep-dortmund.org

Geschäftsführung:\\
Kevin Schmidt\\
Karl Schiller\\
Lena Linhoff\\

\section{Erfassung von http-Protokolldaten}

Bei jedem Zugriff eines Nutzers auf das PeP-Internetangebot werden Daten über
diesen Vorgang in einer Protokolldatei gespeichert und verarbeitet.
Im Einzelnen werden über jeden Zugriff folgende Daten gespeichert:
IP-Adresse, Datum und Uhrzeit des Zugriffs, sowie die besuchte URL.
Die vorübergehende Speicherung der IP-Adresse durch das System ist notwendig,
um eine Auslieferung der Website an den Rechner des Nutzers zu ermöglichen.
Hierfür muss die IP-Adresse des Nutzers für die Dauer der Sitzung gespeichert
bleiben.
Die Speicherung in Logfiles erfolgt, um die Funktionsfähigkeit der Website
sicherzustellen. Eine Auswertung der Daten zu statistischen oder
Marketingzwecken findet in diesem Zusammenhang ebenso wenig statt wie eine
Speicherung dieser Daten zusammen mit anderen personenbezogenen Daten des
Nutzers.

Rechtsgrundlage für die vorübergehende Speicherung der Daten und der Logfiles
ist Art. 6 Abs. 1 lit. f DSGVO. Die Verarbeitung dient der Wahrung eines
berechtigten Interesses unseres Vereins oder eines Dritten. Dieses Interesse
überwiegt die Interessen, Grundrechte und Grundfreiheiten des Betroffenen
nicht.

\section{Cookies}

Unsere Website nutzt Cookies lediglich zur Mitglieder- und
Veranstaltungsregistrierung.
In keinem Fall werden die von uns erfassten Daten an Dritte weitergegeben
oder eine Verknüpfung mit personenbezogenen Daten hergestellt.
Nach Beenden der Browsersitzung werden die Cookies von uns nicht weiter
benötigt und daher anschließend gelöscht.

Natürlich können Sie unsere Website grundsätzlich auch ohne Cookies betrachten.
Internet-Browser sind regelmäßig so eingestellt, dass sie Cookies akzeptieren.
Im Allgemeinen können Sie die Verwendung von Cookies jederzeit über die
Einstellungen Ihres Browsers deaktivieren.
Bitte verwenden Sie die Hilfefunktionen Ihres Internetbrowsers, um zu
erfahren, wie Sie diese Einstellungen ändern können.
Bitte beachten Sie, dass einzelne Funktionen unserer Website möglicherweise
nicht funktionieren, wenn Sie die Verwendung von Cookies deaktiviert haben.

\section{Mitgliedschaft bei PeP et al.}

Auf unserer Internetseite bieten wir Nutzern die Möglichkeit, sich unter
Angabe personenbezogener Daten als PeP-Mitglied anzumelden.
Die Daten werden dabei in eine Eingabemaske eingegeben, mit einer
verschlüsselten SSL-Verbindung an uns übermittelt und in unserer
Mitgliederdatenbank gespeichert.\todo[inline]{wie genau läuft das bei uns?}
Im Rahmen des Anmeldeprozesses wird eine Einwilligung des Nutzers zur
Verarbeitung dieser Daten eingeholt und auf diese Datenschutzerklärung
verwiesen.
Die Daten werden für die Mitgliederverwaltung und -buchhaltung, den Versand
von Mitgliedsunterlagen und vereinsinternen Briefen bzw. E-Mails über
Vereinsaktivitäten verwendet.
Zudem werden die Daten aus der Mitgliederdatenbank mit anderen internen
Datenbanken verknüpft.

Rechtsgrundlage für die Verarbeitung der Daten ist bei Vorliegen einer
Einwilligung des Nutzers Art. 6 Abs. 1 lit. a DSGVO.

Grundsätzlich werden Daten gelöscht, sobald sie für die Erreichung des
Zweckes ihrer Erhebung nicht mehr erforderlich sind.
Dies ist für die während des Anmeldevorgangs zu einer Mitgliedschaft oder zur
Durchführung vorbereitender Maßnahmen dann der Fall, wenn die Daten für die
Durchführung der Mitgliedschaft nicht mehr erforderlich sind.
Auch nach Abschluss oder Beendigung der Mitgliedschaft kann eine
Erforderlichkeit, personenbezogene Daten des Mitglieds zu speichern,
bestehen, um vertraglichen oder gesetzlichen Verpflichtungen nachzukommen.

Als PeP-Mitglied haben Sie jederzeit die Möglichkeit, Ihre Mitgliedschaft
schriftlich oder elektronisch durch eine Austrittserklärung zu beenden.
Die über Sie gespeicherten Daten können Sie zudem via
Online-Änderungsformular abändern lassen.

\section{Veröffentlichung der Kontaktdaten von Funktionsträgern}

PeP-Mitglieder sowie in bestimmten Fällen auch Nicht-PeP-Mitglieder werden
zur Ausübung unterschiedlicher ehrenamtlicher Funktionen gewählt.
Die im Zuge der Wahl übermittelten personenbezogenen Daten werden von PeP et
al. auf der Vereinshomepage gespeichert und veröffentlicht.

Rechtsgrundlage für die Verarbeitung der Daten ist bei Vorliegen einer
Einwilligung des Nutzers Art. 6 Abs. 1 lit. a DSGVO.

Sobald die Amtszeit abgelaufen ist und eine Nachfolge auf der Webseite
eingetragen wurde, werden die Daten automatisch von der Webseite entfernt und
befinden sich nur noch im Verlauf des zugrunde liegenden github Repositories.
Auf Wunsch löschen wir Ihre Daten aus der Versionshistorie des Repositories.

Als betroffene Person haben Sie jederzeit die Möglichkeit, der
Veröffentlichung Ihrer personenbezogenen Daten auf der Vereinshomepage zu
widersprechen. Die über Sie gespeicherten Daten können Sie jederzeit
schriftlich oder elektronisch abändern lassen.

\section{Jahresrückblicke und Protokolle}

PeP et al. e.V. versendet jährlich einen Jahresrückblick, sowie die
Protokolle der Mitgliederversammlungen an alle seine Mitglieder. Zur
transparenten Darstellung unserer Arbeit werden diese Dokumente auf der
Vereinshomepage veröffentlicht.
In diesen Rückblicken und Protokollen werden personenbezogene Daten
veröffentlicht.

Rechtsgrundlage für die Verarbeitung der Daten ist bei Vorliegen einer
Einwilligung des Nutzers Art. 6 Abs. 1 lit. a DSGVO.

Unabhängig von Ihrer Mitgliedschaft bei PeP et al. haben Sie jederzeit die
Möglichkeit, den Bezug des Jahresrückblicks schriftlich oder via
Online-Änderungsformular zu beenden.
Zudem können Sie der Veröffentlichung Ihrer personenbezogenen Daten für die
Zukunft widersprechen.

\section{Online-Formulare zur Anmeldung zu Vereinsaktivitäten und -veranstaltungen}

Auf unserer Internetseite bieten wir Nutzern die Möglichkeit, sich unter
Angabe personenbezogener Daten für Veranstaltungen und Vereinsaktivitäten
anzumelden.
Die Daten werden dabei in eine Eingabemaske eingegeben, mit einer
verschlüsselten SSL-Verbindung an uns übermittelt und gespeichert.
\todo[inline]{Wie genau ist hier unser Ablauf?}
Bei der Organisation und Durchführung bestimmter Veranstaltungen und
Aktivitäten des Vereins übermittelt PeP et al. Daten an beauftragte
Dienstleister oder ehrenamtlich Tätige.
Die beauftragten Dienstleister oder ehrenamtlich Tätigen werden mittels
Vereinbarungen zum Schutz Ihrer personenbezogenen Daten verpflichtet.
Im Rahmen des Anmeldeprozesses wird eine Einwilligung des Nutzers zur
Verarbeitung dieser Daten eingeholt und auf diese Datenschutzerklärung
verwiesen.
Eine Registrierung des Nutzers ist für das Bereithalten bestimmter Inhalte
und Leistungen auf unserer Website sowie zur Erfüllung von Vereinszwecken
oder zur Organisation und Durchführung von Vereinsaktivitäten erforderlich.

Rechtsgrundlage für die Verarbeitung der Daten ist bei Vorliegen einer
Einwilligung des Nutzers Art. 6 Abs. 1 lit. a DSGVO.

Wir löschen Ihre Daten, sobald sie für die Erreichung des Zweckes ihrer
Erhebung nicht mehr erforderlich sind und verpflichten ebenfalls die von uns
beauftragten Dienstleister oder ehrenamtlich Tätigen mittels Vereinbarung zur
Löschung.
Dies ist für die während des Anmeldevorgangs erhobenen Daten der Fall, wenn
die Anmeldung aufgehoben oder abgeändert wird. Auch nach Erreichung des
Zweckes ihrer Erhebung kann eine Erforderlichkeit, personenbezogene Daten der
betroffenen Person zu speichern, bestehen, um gesetzlichen Verpflichtungen
nachzukommen.

Als Nutzer haben Sie jederzeit die Möglichkeit, die Anmeldung zu
Vereinsaktivitäten oder -veranstaltungen schriftlich oder elektronisch
aufzulösen. Die über Sie gespeicherten Daten können Sie jederzeit schriftlich
oder elektronisch abändern lassen.

\section{Kontakt-Formulare und E-Mail-Kontakt}

Treten Sie per E-Mail oder Kontaktformular mit uns in Kontakt, werden die von
Ihnen gemachten Angaben zum Zwecke der Bearbeitung der Anfrage sowie für
mögliche Anschlussfragen gespeichert.
Bei Verwendung des Kontaktformulars wird im Rahmen des Absendevorgangs Ihre
Einwilligung zur Verarbeitung der Daten eingeholt und auf diese
Datenschutzerklärung verwiesen.
Alternativ ist eine Kontaktaufnahme über die bereitgestellte E-Mail-Adresse
möglich. In diesem Fall werden die mit der E-Mail übermittelten
personenbezogenen Daten des Nutzers gespeichert.
Die Daten werden ausschließlich für die Verarbeitung der Konversation
verwendet.
Die sonstigen während des Absendevorgangs verarbeiteten personenbezogenen
Daten dienen dazu, einen Missbrauch des Kontaktformulars zu verhindern und
die Sicherheit unserer informationstechnischen Systeme sicherzustellen.

Rechtsgrundlage für die Verarbeitung der Daten ist bei Vorliegen einer
Einwilligung des Nutzers Art. 6 Abs. 1 lit. a DSGVO. Rechtsgrundlage für die
Verarbeitung der Daten, die im Zuge einer Übersendung einer E-Mail
übermittelt werden, ist Art. 6 Abs. 1 lit. f DSGVO.
Zielt der E-Mail-Kontakt auf den Abschluss eines Vertrages ab, so ist
zusätzliche Rechtsgrundlage für die Verarbeitung Art. 6 Abs. 1 lit. b DSGVO.

Der Nutzer hat jederzeit die Möglichkeit, seine Einwilligung zur Verarbeitung
der personenbezogenen Daten schriftlich oder elektronisch zu widerrufen.
Nimmt der Nutzer per E-Mail Kontakt mit uns auf, so kann er der Speicherung
seiner personenbezogenen Daten jederzeit widersprechen.
In einem solchen Fall kann die Konversation nicht fortgeführt werden.
Alle personenbezogenen Daten, die im Zuge der Kontaktaufnahme gespeichert
wurden, werden in diesem Fall gelöscht.

\section{Versand von Newslettern und Mailings}

Auf unserer Internetseite bieten wir PeP-Mitgliedern und an der PeP et al.
Interessierten die Möglichkeit, sich unter Angabe personenbezogener Daten für
den Empfang von Newslettern oder Mailings anzumelden.
Im Zuge Ihrer Anmeldung als PeP-Mitglied auf der Webseite oder mit Hilfe des
entsprechenden Formulars besteht die Möglichkeit, den Newsletter, die
Jahresrückblicke, Mailings über PeP-Aktivitäten sowie weitere Mitteilungen zu
beziehen.
Dabei werden die E-Mail-Adressen, die die Nutzer dem Verein übermittelt
haben, für die Versendung verwendet. Für die Verarbeitung der Daten wird im
Rahmen des Anmeldevorgangs Ihre Einwilligung eingeholt und auf diese
Datenschutzerklärung verwiesen.
Die Erhebung der E-Mail-Adresse des Nutzers dient dazu, den Newsletter
zuzustellen. Die Erhebung sonstiger personenbezogener Daten im Rahmen des
Anmeldevorgangs dient dazu, einen Missbrauch der Dienste oder der verwendeten
E-Mail-Adresse zu verhindern.

Rechtsgrundlage für die Verarbeitung der Daten nach Anmeldung zum Newsletter
durch den Nutzer ist bei Vorliegen einer Einwilligung des Nutzers Art. 6 Abs.
1 lit. a DSGVO.

Die Daten werden gelöscht, sobald sie für die Erreichung des Zweckes ihrer
Erhebung nicht mehr erforderlich sind. Die E-Mail-Adresse des Nutzers wird
demnach solange gespeichert, wie das Abonnement des Newsletters bzw. seine
Mitgliedschaft bei PeP et al. aktiv ist, soweit nicht vertragliche oder
gesetzliche Verpflichtungen einer Löschung entgegenstehen.

Die Abonnements der Newsletter können jeweils durch den betroffenen Nutzer
jederzeit schriftlich, mittels Online-Änderungsformular oder mithilfe eines entsprechenden Links im Newsletter gekündigt werden.

\begin{comment}
% könnte man mal drüber nachdenken
\section{Online-Spendenformular}

Auf unserer Internetseite bieten wir Nutzern die Möglichkeit, online Spenden vorzunehmen. Nimmt ein Nutzer diese Möglichkeit wahr, werden die in das zugehörige Formular eingegebenen Daten an uns übermittelt und gespeichert. Das Formular wird von der Bank für Sozialwirtschaft AG („BFS“) bereitgestellt. Die eingegebenen Daten werden zur Ausführung des Spendenauftrags daher unmittelbar mit einer verschlüsselten SSL-Verbindung an die der BFS sowie die von der BFS zur Bereitstellung des Formulars eingesetzten technischen Dienstleister weitergegeben. Eine Weitergabe der Daten an sonstige Dritte findet nicht statt. Folgende Daten werden mit dem Formular erhoben: Vollständiger Name (Nachname, Vorname) mit Anrede (optional Titel und Firmenbezeichnung); Anschrift (Straße, Hausnummer, Ort, Postleitzahl, Land); E-Mail-Adresse; Bankdaten (IBAN); Spendendaten (Spendenempfänger, Betrag, Spenden-/ Verwendungszweck, Spendenquittung gewünscht). Zusätzlich bei Spenden über Kreditkarten: Kartentyp, Kartennummer, CVV-/CVC- Prüfnummer, Gültigkeitszeitraum der Kreditkarte. Wird eine Spendenquittung gewünscht, verarbeiten wir die Daten, um eine entsprechende Spendenquittung auszustellen und zuzusenden.

Die erhobenen Daten sind zur Aus- und Durchführung des Spendenauftrags erforderlich. Die E-Mail-Adresse des Nutzers wird benötigt, um den Eingang des Spendenauftrags zu bestätigen. Für weitere Zwecke werden die Daten nicht verwendet. Rechtsgrundlage für die Verarbeitung der Daten ist Art. 6 Abs. 1 lit. b DSGVO.

Im Zeitpunkt der Absendung des Formulars wird zudem die IP-Adresse des Nutzers gespeichert. Die IP-Adresse verwenden wir, um einen Missbrauch des Spendenformulars zu verhindern. Die IP-Adresse wird zum Zweck der Betrugsprävention genutzt und um unberechtigte Transaktionen zum Schaden Dritter zu verhindern. Rechtgrundlage für die Verarbeitung der IP-Adresse ist Art. 6 Abs. 1 lit. f DSGVO.

Die Daten werden gelöscht, sobald sie für die Erreichung des Zweckes ihrer Erhebung nicht mehr erforderlich sind. Bei den Bankdaten ist dies unmittelbar nach Einzug des Spendenbetrags der Fall. Die Adressdaten werden nach ggf. gewünschter Erstellung und Zusendung einer Spendenquittung wie alle weiteren eingegeben Daten im Rahmen steuerrechtlicher Aufbewahrungspflichten gespeichert, dabei jedoch für jegliche andere Verwendung gesperrt. Die während des Absendevorgangs zusätzlich erhobene IP-Adresse wird spätestens nach einer Frist von sieben Tagen gelöscht.

Der Nutzer hat jederzeit die Möglichkeit, der Verarbeitung der Daten zu widersprechen. Es ist allerdings zu beachten, dass bei einem Widerspruch der Spendenauftrag nicht mehr wie gewünscht ausgeführt werden kann.
\end{comment}

\section{Einbindung externer Links}

Diese Datenschutzerklärung gilt nur für Inhalte der Internetangebote von PeP
et al.. Der Verein verwendet auf seinen Webseiten Links auf externe
Webseiten, deren Inhalte sich nicht auf seinen Servern befinden.
Die externen Inhalte dieser Links wurden beim Setzen der Links geprüft.
Es kann jedoch nicht ausgeschlossen werden, dass die Inhalte von den
jeweiligen Anbietern nachträglich verändert werden.
Sollten Sie bemerken, dass die Inhalte der externen Anbieter gegen geltendes
Recht verstoßen, teilen Sie dies bitte den oben genannten Ansprechpersonen
mit.

\section{Datensicherheit}

Wir treffen nach dem Stand der Technik Vorkehrungen, um Ihre Daten vor
Verlust, Zerstörung, Verfälschung, Manipulation und unberechtigtem Zugriff zu
schützen.
Soweit Ihre Daten bei uns erhoben und erfasst werden, erfolgt deren
Speicherung auf besonders geschützten Servern. Diese sind durch technische
und organisatorische Maßnahmen gegen Verlust, Zerstörung, Zugriff,
Veränderung oder Verbreitung Ihrer Daten durch unbefugte Personen geschützt.
Der Zugriff auf Ihre Daten ist nur wenigen, befugten Personen möglich. Diese
sind für die technische, kaufmännische oder redaktionelle Betreuung der
Server zuständig. Alle unsere Mitarbeiterinnen und Mitarbeiter sind zur
Vertraulichkeit verpflichtet.

\section{Ihre Betroffenenrechte}

Unter den angegebenen Kontaktdaten können Sie jederzeit folgende Rechte
ausüben:

\begin{itemize}
  \item Auskunft über Ihre bei uns gespeicherten Daten und deren Verarbeitung,
  \item Berichtigung unrichtiger personenbezogener Daten,
  \item Löschung Ihrer bei uns gespeicherten Daten,
  \item Einschränkung der Datenverarbeitung, sofern wir Ihre Daten aufgrund
        gesetzlicher Pflichten noch nicht löschen dürfen,
  \item Widerspruch gegen die Verarbeitung Ihrer Daten bei uns und
  \item Datenübertragbarkeit, sofern Sie in die Datenverarbeitung
        eingewilligt haben oder einen Vertrag mit uns abgeschlossen haben.
\end{itemize}

Sofern Sie uns eine Einwilligung erteilt haben, können Sie diese jederzeit
mit Wirkung für die Zukunft widerrufen. Sie können sich jederzeit mit einer
Beschwerde an die für Sie zuständige Aufsichtsbehörde wenden. Ihre zuständige
Aufsichtsbehörde richtet sich nach dem Bundesland Ihres Wohnsitzes, Ihrer
Arbeit oder der mutmaßlichen Verletzung. Eine Liste der Aufsichtsbehörden
(für den nichtöffentlichen Bereich) mit Anschrift finden Sie unter: \url{https://www.bfdi.bund.de/DE/Infothek/Anschriften_Links/anschriften_links-node.html}.

\section{Änderungen der Datenschutzerklärung}

PeP et al. behält sich vor, diese Datenschutzerklärung zu ändern. Die
aktuelle Fassung der Datenschutzerklärung finden Sie stets auf den Webseiten
von PeP et al. unter \url{https://registration.pep-dortmund.org/data_privacy_statement}.

Stand: 26.03.2021

\end{document}
