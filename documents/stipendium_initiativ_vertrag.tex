\documentclass[
  paper=a4,
  fontsize=12pt,
  DIV=16,
  parskip=full,
  headinclude=true,
]{scrartcl}

\usepackage{fontspec}
\setsansfont{Source Sans Pro}
\renewcommand{\familydefault}{\sfdefault}

\usepackage{polyglossia}
\setmainlanguage{german}

\usepackage{microtype}

\usepackage[margin=2cm, bottom=3cm]{geometry}

\usepackage{scrlayer-scrpage}
\pagestyle{scrheadings}
\setkomafont{pagehead}{\normalfont}
\setkomafont{pagefoot}{\normalfont\footnotesize}

\usepackage{titling}
\usepackage{booktabs}

\usepackage{xcolor}
\usepackage{calc}
\usepackage[colorlinks=true,urlcolor=blue!50!black]{hyperref}

\usepackage{siunitx}
\usepackage{enumerate}

\renewcommand*{\sectionformat}{\S\thesection\autodot\enskip}

\date{02. Februar 2015}

\begin{document}
  \begin{minipage}{0.45\textwidth}%
	  \large\bfseries PeP et al.\ e.\,V.\\%
	  Begabtenförderungsprogramm –\\%
	  Initiativstipendium%
  \end{minipage}%
  \hfill%
  \begin{minipage}{0.45\textwidth}%
	\hfill\includegraphics[height=1.5cm]{pep.pdf}
  \end{minipage}%
\section*{Stipendiumsvertrag}

Der Verein

Physikstudierende und ehemalige Physikstudierende\\
der TechnischenUniversität Dortmund et al. e.V.\\
Otto-Hahn-Str.~4\\
44221 Dortmund

vertreten durch den ersten Vorsitzenden Henning Moldenhauer nachfoldend
PeP et al. genannt, vergibt an

Herrn/Frau\\
Geboren am\\
wohnhaft in

nachfolgend Stipendiat/in genannt ein Stipendium für sein/ihr
Forschungspraktikum/Auslandssemester/berufsvorbereitendes Praktikum oder
weitergehende wissenschaftliche Qualifikation.
PeP et al.\ und der/die Stipendiat/in treffen folgende Vereinbarung über die
Förderung:

\section{Förderungszweck}

\begin{enumerate}[\qquad(1)]
	\item Der Verein PeP et al. gewährt dem/der Stipendiaten/in nach
		Maßgabe der folgenden Regelungen ein Stipendium, das
		dem/der Stipendiaten/in sein/ihr Forschungspraktikum/
		Auslandssemester/berufsvorbereitendes Praktikum/weitergehende
		wissenschaftliche Qualifikation ermöglichen soll.
	\item Die Annahme eines Stipendiums begründet kein Arbeits-, Dienst-
		oder sonstiges Beschäftigungsverhältnis zwischen dem/der
		Stipendiaten/in und PeP et al.. Es ist kein Entgelt im Sinne
		§ 14 Sozialgesetzbuch IV. Weder ist der/die Stipendiat/in
		PeP et al. gegenüber zu einer Arbeitnehmertätigkeit
		noch zu einer bestimmten Gegenleistung verpflicht.
		Das Stipendium ist ein Zuschuss zum Lebensunterhalt und
		keine Gegenleistung für wissenschaftliche Tätigkeit.
\end{enumerate}

\section{Umfang der Förderung}

\begin{enumerate}[\qquad(1)]
	\item Der Stipendiengeber gewährt dem/der Stipendiaten/in ein
		Stipendium in Höhe von \num{000}\,€ pro Monat innerhalb
		des Zeitraumes vom STARTDATUM EINTRAGEN bis zum ENDDATUM
		EINTRAGEN.
	\item Die Zahlung erfolgt monatlich im Voraus durch Überweisung auf
		das nachstehend genannte Bankkonto:

		Kontoinhaber:\\
		Bankinstitut:\\
		IBAN:\\
\end{enumerate}

\section{Pflichten des/der Stipendiaten/in}

\begin{enumerate}[\qquad(1)]
	\item Der/Die Stipendiat/in verpflichtet sich dazu, sich ernsthaft um die
		Erreichung des Förderziels zu bemühen. Etwaige Hinderungsgründe sind
		PeP et al. umgehend zur Kenntnis zu geben.
	\item Der/Die Stipendiat/in informiert PeP et al. unverzüglich, wenn das
		Vorhaben unterbrochen, geändert, vorzeitig abgeschlossen oder
		abgebrochen wird.
	\item Der/Die Stipendiat/in ist verpflichtet PeP et al. umgehend
		Mitteilung machen, falls ihm/ihr von anderer Stelle eine
		Studienförderung zugesprochen wird.
	\item Über den Verlauf und die Ergebnisse der Studien legt der/die
		Stipendiat/in regelmäßig Zeugnis ab. Der/Die Stipendiat/in
		hat bei einer Förderdauer von mehr als sechs Monaten nach Ablauf
		eines halben Jahres gegenüber PeP et al. einen Zwischenbericht
		von maximal 2 DIN-A4-Seiten über den Stand des Projekts zu
		erstatten. Bei kürzerer Förderdauer und nach Abschluss der
		Förderung muss spätestens einen Monat nach Ende der Förderung
		ein Abschlussbericht von minimal 1 DIN-A4-Seite und maximal
		3 DIN-A4-Seiten bei der oben genannten Stelle eingehen. 
	\item Übt ein/eine Stipendiat/in neben der Bearbeitung des wissenschaftlichen
		Vorhabens eine Berufstätigkeit aus, so ist eine Förderung nach diesen
		Richtlinien ausgeschlossen, sofern es sich nicht um eine Tätigkeit von
		geringem Umfang handelt.
		Als Berufstätigkeit von geringem Umfang gilt eine Tätigkeit bis zu zehn
		Stunden wöchentlich. Der/Die Stipendiat/in hat PeP et al. über
		jede Berufstätigkeit zu informieren.
	\item Unterbricht der/die Stipendiat/in sein/ihr wissenschaftliches Vorhaben,
		so unterrichtet er/sie PeP et al. unverzüglich. Die Zahlung des
		Stipendiums ist vom Zeitpunkt der Unterbrechung an auszusetzen.
		\begin{enumerate}[(a)]
			\item  Zeigt der/die Stipendiat/in das Ende der Unterbrechung
				an, wird die Zahlung wieder aufgenommen und die
				Bewilligung um den Zeitraum der Unterbrechung verlängert.
			\item Bei einer Unterbrechung wegen Krankheit oder aus einem
				anderen wichtigen, von dem/der Stipendiaten/in nicht
				zu vertretenden Grund kann das Stipendium bis zu
				vier Wochen fortgezahlt werden. Eine darüber
				hinausgehende, längerfristige Unterbrechung,
				insbesondere Erkrankung ist unverzüglich anzuzeigen.
		\end{enumerate}
\end{enumerate}

\section{Verantwortlicher Hochschullehrer}

\begin{enumerate}[\qquad(1)]
	\item Die Betreuung des Forschungsvorhabens erfolgt durch NAME DES KONTAKTES.
\end{enumerate}

\section{Haftung}

\begin{enumerate}[\qquad(1)]
	\item Ein gesonderter Versicherungsschutz besteht für den/die
		Stipendiaten/in nicht.
		Für schuldhaft verursachte Schäden haftet er/sie selbst.
		Ihm/Ihr obliegt es, für den entsprechenden Versicherungsschutz
		zu sorgen. 
\end{enumerate}

\section{Widerruf der Förderung}

\begin{enumerate}[\qquad(1)]
	\item Die Bewilligung der Förderung kann seitens PeP et al. aufgehoben
		(zurückgenommen bzw. widerrufen) werden. Ein Aufhebungsgrund
		liegt insbesondere vor,
		\begin{enumerate}[(a)]
			\item wenn die Bewilligung auf unrichtigen oder
				unvollständigen Angaben seitens des/der
				Stipendiaten/in beruht.
			\item wenn der/die Stipendiat/in von öffentlichen oder
				privaten Einrichtungen eine finanzielle Förderung
				desselben Vorhabens erhält.
			\item wenn und ab dem Zeitpunkt, zu dem der/die Stipendiat/in
				eine Nebentätigkeit aufnimmt, die mit der Förderung
				nicht vereinbar ist.
			\item wenn der/die Stipendiat/in sich nicht ernsthaft,
				zügig und konzentriert um die Erreichung des
				Förderziels bemüht.
			\item wenn der/die Stipendiat/in im Rahmen des geförderten
				Vorhabens grob gegen die Regeln guter wissenschaftlicher
				Praxis verstoßen hat und dies von PeP et al. in einem
				abgeschlossenen Verfahren nach den Richtlinien
				der DFG zur Sicherung der guten wissenschaftlichen
				Praxis in ihrer jeweils geltenden Fassung geltend
				gemacht worden ist.
			\item wenn der/die Stipendiat/in seine/ihre sonstigen Pflichten
				aus dem Stipendium grob verletzt.
		\end{enumerate}
	\item Im Falle des Widerrufs der Förderung werden alle Zahlungen mit Wirkung
		auf den im Widerruf genannten Zeitpunkt eingestellt.
	\item Der/Die Stipendiat/in ist verpflichtet, im Falle des Widerrufs oder
		bei Beendigung der Förderung aus anderen Gründen alle über den
		Zeitpunkt der Beendigung hinaus an ihn/sie gezahlten Beträge 
		PeP et al. zurückzuerstatten.
\end{enumerate}

\section{Geheimhaltungsverpflichtung}

\begin{enumerate}[\qquad(1)]
	\item Der/Die Stipendiat/in ist verpflichtet, alle ihm/ihr während
		seines/ihres Forschungsvorhabens mit PeP et al. ausgetauschten
		Informationen vertraulich zu behandeln und ohne Absprache
		Dritten nicht zugänglich zu machen.
	\item PeP et al. ist verpflichtet, alle ihm vom Stipendiaten gemachten
		Angaben vertraulich zu behandeln und diese Dritten nicht
		zugänglich zu machen.
\end{enumerate}

\section{Schlussbestimmungen}

\begin{enumerate}[\qquad(1)]
	\item Dieser Vertrag unterliegt dem deutschen Recht.
	\item Alle Änderungen und Ergänzungen dieses Vertrages bedürfen der
		Schriftform.
\end{enumerate}

\vspace{2cm}
\begin{tabular}{@{}p{0.5\textwidth}@{}p{0.47\textwidth}@{}}%
\rule{6cm}{1pt} & \rule{7cm}{1pt} \\
Ort, Datum & Unterschrift (Stipendiat)\\[2cm]
 & \rule{7cm}{1pt} \\
 & Unterschrift (PeP et al.)
\end{tabular}
\end{document}
