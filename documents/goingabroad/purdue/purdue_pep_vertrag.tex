\documentclass[
  paper=a4,
  fontsize=12pt,
  DIV=16,
  parskip=full,
  headinclude=true,
]{scrartcl}

\usepackage{fontspec}
\setsansfont{Source Sans Pro}
\renewcommand{\familydefault}{\sfdefault}

\usepackage{polyglossia}
\setmainlanguage{german}

\usepackage{graphicx}

\usepackage{microtype}

\usepackage[margin=2cm, bottom=3cm]{geometry}

\usepackage{scrlayer-scrpage}
\pagestyle{scrheadings}
\setkomafont{pagehead}{\normalfont}
\setkomafont{pagefoot}{\normalfont\footnotesize}

\usepackage{titling}
\usepackage{booktabs}
\usepackage{csquotes}

\usepackage{xcolor}
\usepackage{calc}
\usepackage[colorlinks=true,urlcolor=blue!50!black]{hyperref}

\usepackage{siunitx}
\usepackage{enumerate}

\renewcommand*{\sectionformat}{\S\thesection\autodot\enskip}

\date{02. Februar 2015}

\newcommand\vorsitzender{Henning Moldenhauer}

% Daten des Stipendiums / des Stipendiaten
\newcommand\stipyear{2018}    % Name
\newcommand\stipendiat{NAME}    % Name
\newcommand\stipgeburt{DATUM}   % Geburtsdatum
\newcommand\stipanschrift{STADT}  % Wohnort
\newcommand\stipstart{STARTDATUM} % Beginn der Förderung
\newcommand\stipende{ENDDATUM}    % Ende der Förderung
\newcommand\stipkontoinhaber{INHABER}    % Kontodaten auf welches die Förderung
\newcommand\stipbic{BIC}          % überwiesen werden soll.
\newcommand\stipiban{IBAN}
\newcommand\stipbetreuung{NAME DES KONTAKTES UND INSTITUT}  % Name des Projektbetreuers und Institut

\begin{document}
  \begin{minipage}{0.45\textwidth}%
    \large\bfseries PeP et al.\ e.\,V.\\%
    Begabtenförderungsprogramm –\\%
    Goind Abroad \stipyear%
  \end{minipage}%
  \hfill%
  \begin{minipage}{0.45\textwidth}%
  \hfill\includegraphics[height=1.5cm]{pep.pdf}
  \end{minipage}%
\section*{Stipendienvertrag}

Der Verein

Physikstudierende und ehemalige Physikstudierende\\
der TechnischenUniversität Dortmund et al. e.V.\\
Otto-Hahn-Str.~4\\
44221 Dortmund

vertreten durch den ersten Vorsitzenden \vorsitzender\ nachfolgend
PeP et al. genannt, vergibt an

Herrn/Frau \stipendiat \\
Geboren am \stipgeburt \\
wohnhaft in \stipanschrift

nachfolgend Stipendiat/in genannt ein Stipendium für das Programm
\enquote{Going Abroad} an der Purdue Universität, West Lafayette in den USA.
PeP et al.\ und der/die Stipendiat/in treffen folgende Vereinbarung über die
Förderung:

\section{Förderungszweck}

\begin{enumerate}[\qquad(1)]
  \item Der Verein PeP et al. gewährt dem/der Stipendiaten/in nach
    Maßgabe der folgenden Regelungen ein Stipendium, das
    dem/der Stipendiaten/in sein/ihr Forschungsvorhaben ermöglichen soll.
    Das Projekt und dessen Ziele werden im Anhang definiert.
  \item Die Annahme eines Stipendiums begründet kein Arbeits-, Dienst-
    oder sonstiges Beschäftigungsverhältnis zwischen dem/der
    Stipendiaten/in und PeP et al.. Es ist kein Entgelt im Sinne
    § 14 Sozialgesetzbuch IV. Weder ist der/die Stipendiat/in
    PeP et al. gegenüber zu einer Arbeitnehmertätigkeit
    noch zu einer bestimmten Gegenleistung verpflicht.
    Das Stipendium ist ein Zuschuss zur Teilnahme am Programm
    \enquote{Going Abroad \stipyear} und keine Gegenleistung für
    wissenschaftliche Tätigkeit.
\end{enumerate}

\section{Umfang der Förderung}

\begin{enumerate}[\qquad(1)]
  \item Der Stipendiengeber gewährt dem/der Stipendiaten/in im Zuge des
    Stipendiums innerhalb des Zeitraumes vom \stipstart\ bis zum \stipende\
    die nachstehend genannten, maximalen Zuschüsse zur zweckgebundenen Verwendung:

    \begin{tabular}{l r}
      \toprule
      Leistung & Betrag \\
      \midrule
      Reisekosten: Hin-/Rückreise Purdue Universität & \num{1000}\,€ \\
      Beantragung des J1-Visums & komplett \\
      Beantragung Reisepass im Eilverfahren & komplett \\
      Reisekosten: Hin-/Rückreise US Botschaft im Zuge des Visumsverfahrens &
        \num{100}\,€ \\
      \bottomrule
    \end{tabular}
  \item Die Zahlung erfolgt nach durch den/die Stipendiat/in eingereichten
    Belege durch Überweisung auf das nachstehend genannte Bankkonto:

    Kontoinhaber: \stipkontoinhaber \\
    IBAN: \stipiban \\
    BIC: \stipbic
\end{enumerate}

\section{Pflichten des/der Stipendiaten/in}

\begin{enumerate}[\qquad(1)]
  \item Der/Die Stipendiat/in verpflichtet sich dazu, sich ernsthaft um die
    Erreichung des Förderziels zu bemühen. Etwaige Hinderungsgründe sind
    PeP et al. umgehend zur Kenntnis zu geben.
  \item Der/Die Stipendiat/in informiert PeP et al. unverzüglich, wenn das
    Vorhaben unterbrochen, geändert, vorzeitig abgeschlossen oder
    abgebrochen wird.
\end{enumerate}

\section{Verantwortlicher Hochschullehrer}

\begin{enumerate}[\qquad(1)]
  \item Die Betreuung des Forschungsvorhabens erfolgt durch \stipbetreuung\ ,
  sowie Dr.~Andreas~Jung vor Ort an der Purdue Universität.
\end{enumerate}

\section{Haftung}

\begin{enumerate}[\qquad(1)]
  \item Ein gesonderter Versicherungsschutz besteht für den/die
    Stipendiaten/in nicht.
    Für schuldhaft verursachte Schäden haftet er/sie selbst.
    Ihm/Ihr obliegt es, für den entsprechenden Versicherungsschutz
    zu sorgen.
\end{enumerate}

\section{Widerruf der Förderung}

\begin{enumerate}[\qquad(1)]
  \item Die Bewilligung der Förderung kann seitens PeP et al. aufgehoben
    (zurückgenommen bzw. widerrufen) werden. Ein Aufhebungsgrund
    liegt insbesondere vor,
    \begin{enumerate}[(a)]
      \item wenn die Bewilligung auf unrichtigen oder
        unvollständigen Angaben seitens des/der
        Stipendiaten/in beruht.
      \item wenn und ab dem Zeitpunkt, zu dem der/die Stipendiat/in
        eine Nebentätigkeit aufnimmt, die mit der Förderung
        nicht vereinbar ist.
      \item wenn der/die Stipendiat/in im Rahmen des geförderten
        Vorhabens grob gegen die Regeln guter wissenschaftlicher
        Praxis verstoßen hat und dies von PeP et al. in einem
        abgeschlossenen Verfahren nach den Richtlinien
        der DFG zur Sicherung der guten wissenschaftlichen
        Praxis in ihrer jeweils geltenden Fassung geltend
        gemacht worden ist.
      \item wenn der/die Stipendiat/in seine/ihre sonstigen Pflichten
        aus dem Stipendium grob verletzt.
    \end{enumerate}
  \item Im Falle des Widerrufs der Förderung werden alle Zahlungen mit Wirkung
    auf den im Widerruf genannten Zeitpunkt eingestellt.
  \item Der/Die Stipendiat/in ist verpflichtet, im Falle des Widerrufs oder
    bei Beendigung der Förderung aus anderen Gründen alle über den
    Zeitpunkt der Beendigung hinaus an ihn/sie gezahlten Beträge
    PeP et al. zurückzuerstatten.
  \item Im Falle grober Verletzungen der nach §3 definierten Pflichten
    des/der Stipendiaten/in ist der/die Stipendiat/in zur Rückzahlung
    bereits geleisteter Zahlungen verpflichtet.
\end{enumerate}

\section{Geheimhaltungsverpflichtung}

\begin{enumerate}[\qquad(1)]
  \item Der/Die Stipendiat/in ist verpflichtet, alle ihm/ihr während
    seines/ihres Forschungsvorhabens mit PeP et al. ausgetauschten
    Informationen vertraulich zu behandeln und ohne Absprache
    Dritten nicht zugänglich zu machen.
  \item PeP et al. ist verpflichtet, alle ihm vom Stipendiaten gemachten
    Angaben vertraulich zu behandeln und diese Dritten nicht
    zugänglich zu machen.
\end{enumerate}

\newpage
\section{Schlussbestimmungen}

\begin{enumerate}[\qquad(1)]
  \item Dieser Vertrag unterliegt dem deutschen Recht.
  \item Alle Änderungen und Ergänzungen dieses Vertrages bedürfen der
    Schriftform.
\end{enumerate}

\vspace{2cm}
\begin{tabular}{@{}p{0.5\textwidth}@{}p{0.47\textwidth}@{}}%
\rule{6cm}{1pt} & \rule{7cm}{1pt} \\
Ort, Datum & Unterschrift (Stipendiat/in)\\[2cm]
 & \rule{7cm}{1pt} \\
 & Unterschrift (PeP et al.)
\end{tabular}
\end{document}
