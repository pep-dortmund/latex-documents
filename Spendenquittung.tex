\documentclass[fontsize=12pt, paper=a4, DIV14, parskip]{scrartcl}

\usepackage[ngerman]{babel}
\usepackage[utf8]{inputenc}
\usepackage{eurosym}
\DeclareUnicodeCharacter{20AC}{\euro}
\usepackage[T1]{fontenc}
\usepackage{graphicx}

\usepackage[default]{sourcesanspro}
\usepackage[T1]{fontenc}

\usepackage{fancyhdr}
\pagestyle{fancy}
\renewcommand{\headrulewidth}{0pt}

\usepackage{tabu}
\usepackage{array}
\newcolumntype{C}[1]{>{\centering\arraybackslash}m{#1}}
\usepackage{color}
\definecolor{grau}{rgb}{0.830,0.865,0.857}
\usepackage{colortbl}

% ***************************************************
% KOPF- UND FUßZEILE 
% ***************************************************
\lhead{{\bfseries\Large PEP et al. e.V.}\\
	Bestätigung über Geldzuwendungen}

\rhead{\vspace{0.5cm}\parbox{8cm}{\includegraphics[width=7cm]{peplogo.pdf}}}

\lfoot{\fontsize{9}{9} \selectfont
       \begin{minipage}{5cm}
       \raggedright
       PEP et al. e.V.\\
       www.pep-dortmund.org\\
       \vspace{16pt}
       \end{minipage}}

\cfoot{\fontsize{9}{9} \selectfont
       \begin{minipage}{3cm}
       \raggedright
       Christophe Cauet\\
       c/o PEP et al. e.V.\\
       Solbergweg 86\\
       44225 Dortmund
       \end{minipage}}

\rfoot{\fontsize{9}{9} \selectfont
       \begin{minipage}{5cm}
       \raggedright
       Bankverbindung\\
       Dortmunder Volksbank\\
       IBAN: DE22 4416 0014 6348 4161 00\\
       BIC: GENODEM1DOR
       \end{minipage}}

\begin{document}
\parbox{20cm}{\hspace{1cm}}
\parbox{20cm}{\hspace{1cm}}

Aussteller

Physikstudierende und ehemalige Physikstudierende an der TU Dortmund et al. e.V.\\
Henning Moldenhauer\\
Holsteiner Straße 33\\
44145 Dortmund

Bestätigung über Geldzuwendungen im Sinne § 10b des Einkommensteuergesetzes an eine der in § 5 Abs. 1 Nr. 9 des Körperschaftsteuergesetzes bezeichneten Körperschaften, Personenvereinigungen oder Vermögensmassen.

Name und Anschrift des Zuwendenden

DEG Dach-Fassade Holz eG\\
Frau Edith Weerd\\ 
Oberster Kamp 6\\
59069 Hamm
\renewcommand{\arraystretch}{1.5}
\begin{table}[h!]
	\centering
	\begin{tabu}{|[1pt]C{4.5cm}|[1pt]C{6cm}|[1pt]C{4.5cm}|[1pt]}
		\tabucline[1pt]{-}
		\rowcolor{grau}
		{\bfseries \scshape Betrag in Ziffern}	&	{\bfseries \scshape Betrag in Buchstaben}	&	{\bfseries \scshape Tag der Zuwendung}\\
		\tabucline[1pt]{-}
		\rmfamily 4.000,-- €	&	Viertausend Euro	&	05.05.2013\\
		\tabucline[1pt]{-}
	\end{tabu}
\end{table}
\vspace{-.7cm}

{\small
Wir sind wegen Förderung der Volks- und Berufsbildung einschließlich der Studentenhilfe nach dem letzten uns zugegangenen Freistellungsbescheid des Finanzamtes Dortmund-Ost, StNr. 317/5941/4882, vom 16.07.2012 als gemeinnützig anerkannt und für die Jahre 2009 bis 2011 nach §5 Abs. 1 Nr. 9 des Körperschaftsteuergesetzes von der Körperschaftsteuer befreit.
\vspace{-.2cm}

Es wird bestätigt, dass die Zuwendung nur zur Förderung der Volks- und Berufsbildung einschließlich der Studentenhilfe (im Sinne der Anlage 1 – zu §48 Abs. 2 Einkommensteuer-Durchführungsverordnung – Abschnitt A Nr. 4) verwendet wird.}

Dortmund, den \today

\vspace{20pt}

Henning Moldenhauer\\
Vorstand PeP et al. e.V.

{\fontsize{10}{10} \selectfont
Hinweis:

Wer vorsätzlich oder grob fahrlässig eine unrichtige Zuwendungsbestätigung erstellt oder wer veranlasst, dass Zuwendungen nicht zu den in der Zuwendungsbestätigung angegebenen steuerbegünstigten Zwecken verwendet werden, haftet für die Steuer, die dem Fiskus durch einen etwaigen Abzug der Zuwendungen bei Zuwendenden entgeht (§10b Abs. 4 EStG, §9 Abs. 3 KStG, §9 Nr. 5 GewStG).
\newline
Diese Bestätigung wird nicht als Nachweis für die steuerliche Berücksichtigung der Zuwendung anerkannt, wenn das Datum des Freistellungsbescheids länger als 5 Jahre bzw. das Datum der vorläufigen Bescheinigung länger als 3 Jahre seit Ausstellung der Bestätigung zurückliegt (BMF vom 15.12.1994 – BStBI I S. 884)

}

\end{document}
