\documentclass[
  pepbrief,
  fontsize=12pt,
  paper=a4,
  DIV=14,
  parskip=half,
  backaddress=false,
]{scrlttr2}

\usepackage{fontspec}
\usepackage{polyglossia}
\setmainlanguage{german}

\usepackage{microtype}
\usepackage{graphicx}

\usepackage{tabu}
\usepackage{array}
\usepackage{colortbl}
\usepackage{siunitx}

\newcolumntype{C}[1]{>{\centering\arraybackslash}m{#1}}

\usepackage{xcolor}
\usepackage[colorlinks=true,urlcolor=blue!50!black]{hyperref}
\definecolor{grau}{rgb}{0.830,0.865,0.857}

\author{Netter PhysiKon Helfer}
\setkomavar{subject}{Bestätigung Ihrer Anmeldung zur PhysiKon 2023}

\begin{document}
\begin{letter}{%
  Frau Musterfrau \\
  Musterstraße 1 \\
  12345 Musterhausen
}
% ANREDE
\opening{Sehr geehrte Frau Musterfrau,}
% BRIEFTEXT
hiermit bestätigen wir Ihre Anmeldung zur PhysiKon 2023 (Onlinevortrag im Zeitraum 17.-19.04.23, 
Präsenzmesse an der TU Dortmund am 20.04.23). 
Wir freuen uns sehr Sie als Aussteller begrüßen zu dürfen!

Im Folgenden erhalten Sie alle wichtigen Informationen zur Durchführung der Messe. Weiter unten finden Sie 
die Zahlungsinformationen für den Ausstellerbeitrag. 

\begin{itemize}
  \item \textbf{Messestand:} Sie erhalten eine Ausstellungsfläche von 2,5m x 2,5m.
    Auf Wunsch kann gerne ein Stehtisch, ein Ablagetisch und/oder eine Stellwand 
    zur Verfügung gestellt werden.
    Für eine ausreichende Stromversorgung (500 Watt), sowie einen Internetzugang wird gesorgt.
    Bei weiteren Fragen zur Gestaltung Ihres Standes kommen Sie gerne auf uns zu.
    \item \textbf{Messekatalog:} Zur PhysiKon wird ein digitaler Messekatalog erscheinen, der auf unserer Website 
    einzusehen sein wird.
    Für Ihre Katalogseite benötigen wir folgende Informationen Ihres Unternehmens:
    \begin{itemize}
      \item Adresse
      \item Kontakt (Ansprechpartner, Email-Adresse) für Bewerbungen
      \item Branche
      \item Standorte
      \item Anzahl der Mitarbeiter
      \item Gewünschte Zusatzqualifikationen
    \end{itemize}
    Weiterhin benötigen wir einen kurzen Text (max. 500 Zeichen), der Ihr Unternehmen vorstellt, 
    und ein Anzeigenbild, 
    das auf der Seite erscheinen soll.
    \item \textbf{Onlinevortrag:} Im Vorfeld zur Berufsmesse bieten wir Ihnen die Möglichkeit sich im Rahmen eines einstündigen Vortrags interessierten
    Studierenden vorzustellen. Um die Einrichtung des Zoom-Meetings kümmern wir uns. 
    \item \textbf{Jobbörse:} Auf unserer \href{https://pep-dortmund.org/jobboerse/}{Jobbörse} können 
    Sie offene Stellen, Praktika oder Traineeprogramme veröffentlichen. Damit die Anzeigen möglichst 
    aktuell sind, werden wir Sie einige Wochen vor der Veranstaltung bitten uns Stellenangebote zuzusenden. 
    Sie können die Jobbörse aber auch gerne schon ab sofort nutzen.
    \item \textbf{Catering:} Für das leibliche Wohl des Standpersonals wird selbstverständlich von uns gesorgt.
    \item \textbf{Veranstaltungsort:}
    \vspace{3mm}\\
    Emil-Figge-Str. 50\\
    Foyer \\
    44227 Dortmund
    \item \textbf{Parkplätze:} Der Parkplatz unterhalb des Mensagebäudes der TU Dortmund befindet sich nahe dem Veranstaltungsort und bietet  
    ausreichende Parkmöglichkeiten. 
    \item \textbf{Anreise mit der Bahn:} Die S-Bahn-Haltestelle "Dortmund Universität" befindet sich in unmittelbarer Nähe zum Veranstaltungsort.
    Am Tag der PhysiKon werden wir den Weg ausreichend beschildern.
    Vom Dortmunder Hauptbahnhof erreichen Sie die TU Dortmund alle 20 Minuten mit der Linie S1 von Gleis 7.
\end{itemize}

Für Fragen, Probleme oder Anregungen sind wir unter \href{mailto:physikon@pep-dortmund.org}{physikon@pep-dortmund.org} zu erreichen. 
Alle Informationen rund um die PhysiKon finden Sie auf \href{https://physikon.pep-dortmund.org/}{physikon.pep-dortmund.org}.

Der Alumni-Verein PeP et al. e.\,V. unterstützt die Fakultät Physik in Forschung und Lehre und fördert den Kontakt und Erfahrungsaustausch zwischen dem Fachbereich,
seinen derzeitigen Studierenden und seinen Absolventen. Neben der PhysiKon veranstalten wir jedes Jahr die Absolventenfeier der Fakultät Physik, 
sowie Workshops, Exkursionen
und die Sommerakademie für Physikstudierende. 
Weitere Informationen über PeP et al. finden Sie unter
\href{https://pep-dortmund.org/}{pep-dortmund.org}.

Hiermit erhalten Sie die Zahlungsinformationen für die oben genannten Leistungen.
\renewcommand{\arraystretch}{1.2}
\begin{center}
  \bfseries\scshape
  \begin{tabular}{| c | c | c |}
    \hline
    \rowcolor{grau}
     Betrag in Ziffern  & Betrag in Buchstaben & Kontoinformationen\\
    \hline
    1300,–\,€ & \normalfont Eintausenddreihundert  Euro & \normalfont \begin{tabular}{@{}c@{}}Pep et al. e.V. \\ DE22\,4416\,0014\,6348\,4161\,00 \\ GENODEM1DOR \end{tabular}\\
    \hline
  \end{tabular}
\end{center}
Wir bitten Sie die Ausstellergebühr von 1300 Euro bis zum 1. Juni 2023 auf unser Konto bei der Dortmunder Volksbank zu überweisen.

Wir freuen uns auf eine erfolgreiche PhysiKon mit Ihnen!

% BRIEFSCHLUSS
\closing{Viele Grüße aus Dortmund}

% ANHANG
\end{letter}
\end{document}
