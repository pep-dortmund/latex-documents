\documentclass[
  pepbrief,
  fontsize=12pt,
  paper=a4,
  DIV=14,
  parskip=half,
  backaddress=false,
]{scrlttr2}

\KOMAoptions{foldmarks=off}
\usepackage{fontspec}
\usepackage{polyglossia}
\setmainlanguage{german}

\usepackage{microtype}
\usepackage{graphicx}

\usepackage{tabu}
\usepackage{array}
\usepackage{colortbl}
\usepackage{siunitx}

\newcolumntype{C}[1]{>{\centering\arraybackslash}m{#1}}

\usepackage{xcolor}
\usepackage[colorlinks=true,urlcolor=blue!50!black]{hyperref}
\definecolor{grau}{rgb}{0.830,0.865,0.857}

\author{Das PhysiKon Team}
\setkomavar{subject}{Bestätigung Ihrer Anmeldung zur PhysiKon 2025}

\begin{document}
\begin{letter}{%
  <Vorname Nachname>\\
  Musterstraße 1 \\
  12345 Musterhausen
}
% ANREDE
\opening{Guten Tag <Vorname Nachname>,}
% BRIEFTEXT
hiermit bestätigen wir Ihre Anmeldung zur PhysiKon 2025 (Onlinevortrag im Zeitraum 14.-16.04.25, 
Präsenzmesse an der TU Dortmund am 17.04.25). 
Wir freuen uns sehr, Sie als Aussteller begrüßen zu dürfen!

Im Folgenden erhalten Sie alle wichtigen Informationen zur Durchführung der Messe:

\begin{itemize}
  \item \textbf{Messestand:} Sie erhalten eine Ausstellungsfläche von 2,5m x 2,5m.
    Auf Wunsch kann gerne ein Stehtisch, ein Ablagetisch und/oder eine Stellwand 
    zur Verfügung gestellt werden.
    Für eine ausreichende Stromversorgung (500 Watt), sowie einen Internetzugang wird gesorgt.
    Bei weiteren Fragen zur Gestaltung Ihres Standes kommen Sie gerne auf uns zu.
    \item \textbf{Messekatalog:} Zur PhysiKon wird ein digitaler Messekatalog erscheinen, der auf unserer Website 
    einzusehen sein wird.
    Für Ihre Katalogseite benötigen wir folgende Informationen Ihres Unternehmens:
    \begin{itemize}
      \item Logo (wenn möglich als Vektorgrafik, .svg, .eps oder .pdf)
      \item kurzer Text (max.\ 500 Zeichen!)
      \item Anzeigenbild (.jpg oder .png)
      \item Adresse
      \item Kontakt (Ansprechpartner*innen, Email-Adresse) für Bewerbungen
      \item Branche
      \item Standorte
      \item Anzahl der Mitarbeiter*innen
      \item Gewünschte Zusatzqualifikationen
      \item Link zu Ihrem Karriereportal (falls vorhanden)
    \end{itemize}
    \item \textbf{Onlinevortrag:} Im Vorfeld zur Berufsmesse bieten wir Ihnen die Möglichkeit sich im Rahmen eines einstündigen Vortrags interessierten
    Studierenden vorzustellen. Um die Einrichtung des Zoom-Meetings kümmern wir uns. 
    \item \textbf{Jobbörse:} Auf unserer \href{https://pep-dortmund.org/jobboerse/}{Jobbörse} können 
    Sie offene Stellen, Praktika oder Traineeprogramme veröffentlichen. Damit die Anzeigen möglichst 
    aktuell sind, werden wir Sie einige Wochen vor der Veranstaltung bitten, uns Stellenangebote zuzusenden. 
    Sie können die Jobbörse aber auch gerne schon vorab nutzen, wenn sie aktuelle Stellenausschreibungen haben.
    Unsere Jobbörse ist unabhängig von der PhysiKon immer online.
    \item \textbf{Catering:} Für das leibliche Wohl des Standpersonals wird selbstverständlich von uns gesorgt.
    \item \textbf{Veranstaltungsort:} Emil-Figge-Str. 50, 44227 Dortmund
    \item \textbf{Parkplätze:} Der Parkplatz unterhalb des Mensagebäudes der TU Dortmund befindet sich nahe dem Veranstaltungsort und bietet  
    ausreichende Parkmöglichkeiten. Wir werden Ihnen kurz vor der Veranstaltung noch mal detaillierte Informationen für die Anfahrt und Anlieferungsparkplätze für sperriges Ausstellungsmaterial zusenden. 
    \item \textbf{Anreise mit der Bahn:} Die S-Bahn-Haltestelle "Dortmund Universität" befindet sich in unmittelbarer Nähe zum Veranstaltungsort.
    Am Tag der PhysiKon werden wir den Weg ausreichend beschildern.
    Vom Dortmunder Hauptbahnhof erreichen Sie die TU Dortmund alle 20 Minuten mit der Linie S1 von Gleis 7.
    \item \textbf{Ausstellerbeitrag:} Die Teilnahme am gesamten Programm der PhysiKon (Onlinevortrag und Messestand) kostet 1300 Euro.
    Der Ausstellerbeitrag für eine Online-Teilnahme (nur Onlinevortrag ohne Messestand) beträgt 800 Euro.
    Die Rechnung über den Ausstellerbeitrag und alle Zahlungsinformationen erhalten Sie einige Wochen vor der Veranstaltung.
    \textbf{Senden Sie uns bitte eine korrekte Rechnungsadresse zu, falls diese von der Unternehmensadresse abweicht.}
\end{itemize}

Für Fragen, Probleme oder Anregungen sind wir jederzeit unter \href{mailto:physikon@pep-dortmund.org}{physikon@pep-dortmund.org} zu erreichen. 
Alle Informationen rund um die PhysiKon finden Sie auf \href{https://physikon.pep-dortmund.org/}{physikon.pep-dortmund.org}.

Der gemeinnützige Alumni-Verein PeP et al. e.\,V. unterstützt die Fakultät Physik in Forschung und Lehre und fördert den Kontakt und Erfahrungsaustausch zwischen dem Fachbereich,
seinen derzeitigen Studierenden und seinen Absolventen.
Weitere Informationen über PeP et al. finden Sie unter
\href{https://pep-dortmund.org/}{pep-dortmund.org}.

Wir freuen uns auf eine erfolgreiche PhysiKon mit Ihnen!

% BRIEFSCHLUSS
\closing{Viele Grüße aus Dortmund}

% ANHANG
\end{letter}
\end{document}
