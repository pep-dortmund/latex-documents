\documentclass[
  pepbrief,
  fontsize=12pt,
  paper=a4,
  DIV=14,
  parskip=half,
  backaddress=false,
]{scrlttr2}

\usepackage{fontspec}
\usepackage{polyglossia}
\setmainlanguage{german}

\usepackage{microtype}
\usepackage{graphicx}

\usepackage{xcolor}
\usepackage[colorlinks=true,urlcolor=blue!50!black]{hyperref}



\author{Lena Linhoff}
\setkomavar{subject}{Bestätigung Ihrer Anmeldung zur PhysiKon 2019}

\begin{document}
\begin{letter}{%
  Firma X\\
  Herr Max Mustermann\\
  Spendenstraße 6\\
  44137 Dortmund%
}
% ANREDE
\opening{Sehr geehrte Frau HR-Perle,}
% BRIEFTEXT
hiermit bestätigen wir Ihre Anmeldung zur PhysiKon am 11. April 2019. Wir freuen uns sehr Sie als Aussteller dabei zu haben!
Hier erhalten Sie alle wichtigen Informationen sowie Zahlungsinformationen:

Für die Ausstellergebühr von 500€ erhalten Sie das folgende Rundum-Sorglos-Paket:

\begin{itemize}
  \item \textbf{Messestand:} Jedes Unternehmen erhält eine Ausstellungsfläche von ca. 2,5 m x 2,5 m.
    Auf Wunsch kann gerne ein Stehtisch oder auch ein Ablagetisch zur Verfügung gestellt werden.
    Für eine ausreichende Stromversorgung, sowie einen Internetzugang wird gesorgt.
    Wir bitten allerdings darum, keine allzu großen Verbraucher (zum Beispiel Beleuchtung für Roll-Up-Banner und Stellwände, etc.) mitzubringen.
    Größere Stellflächen sind aus organisatorischen Gründen nicht möglich.
    \item \textbf{Messekatalog:} Zur PhysiKon wird ein Messekatalog erscheinen, der an die Besucherinnen und Besucher kostenlos verteilt wird.
    Für Ihre Doppelseite im Katalog brauchen wir folgende Kennzahlen Ihres Unternehmens:
    \begin{itemize}
      \item Adresse
      \item Kontakt (Ansprechpartner, Email-Adresse) für Bewerbungen
      \item Branche
      \item Standorte
      \item Anzahl der Mitarbeiter
      \item Gewünschte Zusatzqualifikationen
    \end{itemize}
    Weiterhin benötigen wir einen kurzen Text (max. 1000 (?) Zeichen), der Ihr Unternehmen gut beschreibt, und ein Anzeigenbild (Format DIN A4 oder A5), das auf der linken Seite
    der Doppelseite erscheinen soll.
    \item \textbf{Präsentation:} Den Absolventen und Studierenden sollen auf der PhysiKon nicht nur Unternehmen präsentiert werden, sondern auch die Berufsrealität näher gebracht werden.
    Wir bieten Ihnen daher die Möglichkeit, ihr Unternehmen und Ihren Berufsalltag in einem kurzen Vortrag einem größeren Publikum vorzustellen.
    Falls Sie einen Vortrag anbieten möchten, geben Sie uns bitte Bescheid.
    \item \textbf{Stellenmarkt:} Auf dem Stellenmarkt können Sie offene Stellen, Praktika oder Traineeprogramme veröffentlichen. Damit die Anzeigen möglichst aktuell sind,
    werden wir Sie einige Wochen vor der Veranstaltung bitten uns Stelleangebote zuzusenden.
    \item \textbf{Catering:} Für das leibliche Wohl des Standpersonals wird selbstverständlich von uns gesorgt.
    \item \textbf{Zahlungsinformationen:} Wir bitten Sie die Ausstellergebühr von 500 Euro bis zum 1. Februar 2019 auf unser Konto bei der Dortmunder Volksbank zu überweisen:
    \vspace{3mm}\\
    PeP et al.\\
Dortmunder Volksbank\\
IBAN:\@ \mbox{DE22 4416 0014 6348 4161 00}\\
BIC:\@ GENODEM1DOR%
    \item \textbf{Veranstaltungsort:}
    \vspace{3mm}\\
    Campus Treff\\
    Vogelpothsweg 118\\
    44227 Dortmund
\end{itemize}

Für Fragen, Probleme oder Anregungen sind wir unter physikon@pep-dortmund.org zu erreichen. Alle Informationen finden Sie auch auf www.physikon.pep-dortmund.org.

Der Alumni-Verein PeP et al. e.\,V. unterstützt die Fakultät Physik in Forschung und Lehre und fördert den Kontakt und Erfahrungsaustausch zwischen dem Fachbereich,
seinen derzeitigen Studierenden und seinen Absolventen. Neben der PhysiKon veranstalten wir jedes Jahr die Absolventenfeier der Fakultät Physik, sowie Workshops, Exkursionen
und die Sommerakademie für Physikstudierende. Außerdem finanzieren wir jährlich Stipendien für herausragende Studenten. Weitere Informationen über PeP et al. finden Sie unter
www.pep-dortmund.org.

Wir freuen uns auf eine erfolgreiche PhysiKon mit Ihnen!


% BRIEFSCHLUSS
\closing{Mit freundlichen Grüßen}

% ANHANG
\end{letter}
\end{document}
