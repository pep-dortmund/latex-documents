\documentclass[
  pepbrief,
  fontsize=12pt,
  paper=a4,
  DIV=14,
  parskip=half,
  backaddress=false,
]{scrlttr2}

\usepackage{fontspec}
\usepackage{polyglossia}
\setmainlanguage{german}

\usepackage{microtype}
\usepackage{graphicx}

\usepackage{tabu}
\usepackage{array}
\usepackage{colortbl}
\usepackage{siunitx}

\newcolumntype{C}[1]{>{\centering\arraybackslash}m{#1}}

\usepackage{xcolor}
\usepackage[colorlinks=true,urlcolor=blue!50!black]{hyperref}
\definecolor{grau}{rgb}{0.830,0.865,0.857}

\author{Lena Linhoff}
\setkomavar{subject}{Bestätigung Ihrer Anmeldung zur PhysiKon 2022}

\begin{document}
\begin{letter}{%
  Frau Musterfrau \\
  Musterstraße 1 \\
  12345 Musterhausen
}
% ANREDE
\opening{Sehr geehrte Frau Mustermann,}
% BRIEFTEXT
hiermit bestätigen wir Ihre Anmeldung zur PhysiKon vom 25. - 28. April 2022. Wir freuen uns sehr Sie als Aussteller begrüßen zu dürfen!

Im Folgenden erhalten Sie alle wichtigen Informationen zur Durchführung der Messe:

\begin{itemize}
  \item \textbf{Online-Messekatalog:}
  Auf unserer Homepage stellen wir alle Aussteller kurz vor, damit sich die Besucher vorab oder während der Messe einen Überblick verschaffen können und nach der Messe eine Kontakadresse finden.
  Dafür brauchen wir folgende Kennzahlen Ihres Unternehmens:
  \begin{itemize}
    \item Kontakt (Ansprechpartner, Email-Adresse) für Bewerbungen
    \item Branche
    \item Standort(e)
    \item Anzahl der Mitarbeiter
    \item Gewünschte Zusatzqualifikationen
    \item kurzer Text (ca. 500 Zeichen), der das Unternehmen und die Einsatzgebiete von MINT-Fachkräften gut beschreibt
    \item ein Anzeigenbild in einem Format Ihrer Wahl
    \item ein Logo Ihres Unternehmens
  \end{itemize}
  \item \textbf{Online-Vortrag:} Ihr Zeitslot für den Online-Vortrag beträgt eine Stunde lang.
  Wir nutzen die Software Zoom.
  Falls Sie Zoon in Ihrem Unternehmen nicht verwenden dürfen, besteht die Möglichkeit auf Webex auszuweichen.
  Zoom bietet die Möglichkeit Untergruppenräume anzulegen, sodass Gespräche in kleineren Gruppen parallel stattfinden können.
  Über die genaue Gestaltung Ihrer Stunde und Ihre Anforderungen können Sie uns kurz vorher informieren.
  Falls Sie Zoom nicht kennen, bieten wir gerne eine kurzes Test-Meeting an, in dem Sie sich mit der Software vertraut machen können.
  \item \textbf{Messestand:} Sie erhalten eine Ausstellungsfläche von ca. 2,5 m x 2,5 m.
  Auf Wunsch kann gerne ein Stehtisch oder auch ein Ablagetisch zur Verfügung gestellt werden.
  Für eine ausreichende Stromversorgung (500 Watt), sowie einen Internetzugang wird gesorgt.
  Wir bitten allerdings darum, keine allzu großen Verbraucher (zum Beispiel Beleuchtung für Roll-Up-Banner und Stellwände, etc.) mitzubringen.
  Aufgrund der begrenzten Fläche möchten wir Sie auch bitten keine großen Aufbauten für Ihren Stand mitzubringen.
  Roll-Up-Banner, Prospektständer, etc. sind natürlich kein Problem.
  Wir stellen Ihnen gerne Stellwände zur Verfügung, um Werbematerialien aufzuhängen.
  Größere Stellflächen sind aus organisatorischen Gründen leider nicht möglich.
  \item \textbf{Catering:} Für das leibliche Wohl des Standpersonals wird selbstverständlich von uns gesorgt.
  \item \textbf{Veranstaltungsort:}
  \vspace{3mm}\\
  Emil-Figge Straße 50\\
  44227 Dortmund
  \item \textbf{Anreise mit der Bahn:} Die S-Bahn-Haltestelle "Dortmund Universität befindet sich in unmittelbarer Nähe zum Veranstaltungsort.
  Am Tag der PhysiKon werden wir den Weg zum Veranstaltungsort ausreichend beschildern.
  Vom Dortmunder Hauptbahnhof aus erreichen Sie die TU Dortmund alle 20 Minuten mit der S1 von Gleis 7.
  \item \textbf{Jobbörse:} PeP et al. veröffentlicht auf der Homepage des Vereins \url{https://pep-dortmund.org} Stellenanzeigen von Unternehmen, die MINT-Fachkräfte suchen.
  Senden Sie uns Ihre Stellenanzeigen gerne an \href{mailto:jobboerse@pep-dortmund.org}{jobboerse@pep-dortmund.org}.
  \item \textbf{Rechnung:} Die Rechnung über die Ausstellergebühr erhalten Sie von uns nach der Veranstaltung.
  Bitte geben Sie uns dazu die korrekte Rechnungsadresse Ihres Unternehmens.
\end{itemize}

Für Fragen, Probleme oder Anregungen sind wir unter \href{mailto:physikon@pep-dortmund.org}{physikon@pep-dortmund.org} zu erreichen.
Alle Informationen finden Sie auch auf \url{https://physikon.pep-dortmund.org}.

Der Alumni-Verein PeP et al. e.\,V. unterstützt die Fakultät Physik in Forschung und Lehre und fördert den Kontakt und Erfahrungsaustausch zwischen dem Fachbereich,
seinen derzeitigen Studierenden und seinen Absolventen. Neben der PhysiKon veranstalten wir jedes Jahr die Absolventenfeier der Fakultät Physik, sowie Workshops, Exkursionen
und die Sommerakademie für Physikstudierende. Außerdem finanzieren wir jährlich Stipendien für herausragende Studierende. Weitere Informationen über PeP et al. finden Sie unter
\url{https://pep-dortmund.org}

Wir freuen uns auf eine erfolgreiche PhysiKon mit Ihnen!

% BRIEFSCHLUSS
\closing{Viele Grüße aus Dortmund}

% ANHANG
\end{letter}
\end{document}
