\documentclass[
  paper=a4,
  fontsize=12pt,
  DIV=16,
  headheight=30pt,
  footheight=45pt,
  headinclude,
  parskip=half,
]{scrartcl}

\usepackage{fontspec}
\setsansfont{Source Sans Pro}
\renewcommand{\familydefault}{\sfdefault}
\newfontfamily\geomfont{DejaVu Sans}
\newcommand\checkbox{{\geomfont\symbol{"25A2}}}

\usepackage{polyglossia}
\setmainlanguage{german}

\usepackage{microtype}
\usepackage{graphicx}
\usepackage[table]{xcolor}

\usepackage{scrlayer-scrpage}
\pagestyle{scrheadings}
\setkomafont{pagehead}{\normalfont}
\setkomafont{pagefoot}{\normalfont\footnotesize}
\ohead{\includegraphics[height=1.5cm]{images/logo-pfeil-gruen.pdf}}
\ihead{%
  \large\textbf{Infobrief zur PhysiKon 2021}\\%
}
\cfoot{%
  \parbox[c]{0.3\textwidth}{%
    PeP et al.\ e.\,V.\\
z.\,Hd.\ Marie Schmitz\\
Otto-Hahn-Straße 4a\\
44227 Dortmund%
%
  }%
}

\ofoot{%
  \parbox[c]{0.3\textwidth}{%
%   Bankverbindung\\
  Dortmunder Volksbank\\
IBAN:\@ DE22 4416 0014 6348 4161 00\\
BIC:\@ GENODEM1DOR%
%
  }%
}
\ifoot{%
  \includegraphics[height=1cm]{pep.pdf}\\
  \url{www.pep-dortmund.org}
}


\usepackage{titling}
\usepackage{booktabs}
\usepackage{enumitem}
\usepackage{csquotes}

% \usepackage{xcolor}
\usepackage{calc}
\usepackage[colorlinks=true,urlcolor=blue!50!black]{hyperref}
\renewcommand{\arraystretch}{1.5}

\newcommand\MyTextField[2][]{\TextField[#1, backgroundcolor=black!10, charsize=0pt, borderwidth=0]{#2}}
\renewcommand*{\LayoutTextField}[2]{\makebox[\widthof{#1: }][l]{#1: }%
\raisebox{0.8\baselineskip}{\raisebox{-\height}{#2}}}


\begin{document}

\section*{Herzlich Willkommen zur PhysiKon-Woche 2021!}

Schön, dass Sie bei unserer Jobmesse PhysiKon mit an Board sind!
Dieses jahr wird die Jobmesse sondern online vom 19. bis 23.04.2021 stattfinden.
In diesem Handzettel können wir hoffentlich alle wichtigen Fragen zur Veranstaltung beantworten.
Falls noch Fragen offen sind, melden Sie sich gerne bei uns!

\subsection*{Technische Voraussetzungen}

Die PhysiKon 2021 wird online über das Videokonferenzsystem \textbf{Zoom} stattfinden.
Sie benötigen dafür eine stabile Internetverbindung, einen Internetbrowser (am besten Safari oder Firefox), einen Lautsprecher und Mikrofon oder ein Headset.
Wir werden den Link einige Tage vorher zur Verfügung stellen.
Zoom ist recht intuitiv zu bedienen.
Falls Sie sich mit der Software nicht auskennen und eine Generalprobe durchführen möchten, geben Sie uns einfach Bescheid.
Wir organisieren gerne ein kurzes Test-Meeting mit Ihnen!

Zoom stellt eine \enquote{Breakout-Room}-Funktion zur Verfügung.
So können sich kleinere Gruppen in separaten Räumen unterhalten.
Wir werden diese Funktion anbieten, damit Sie sich zum Beispiel nach Ihrem Vortrag noch mit interessierten BesucherInnen weiter austauschen können, wenn der nächste Vortrag schon startet.
Denkbar wäre auch, dass Sie sich nach einer allgemeinen Vorstellung mit mehreren VertreterInnen Ihres Unternehmens (vielleicht aus verschiedenen Fachbereichen) auf verschiedene Zoom-Räume verteilen.
Dies sei nur als Anregung erwähnt, Sie müssen die Breakout-Funktion selbstverständlich nicht nutzen.

\subsection*{Ablauf und Präsentation}
Sie haben eine Stunde zur Verfügung, die Sie im Grunde frei gestalten können.
Klassischerweise können Sie einen Vortrag halten und Ihre Folien über das Videokonferenzsystem mit den Publikum teilen.
Der Vortrag sollte ungefähr 20 Minuten lang sein, damit genug Zeit für Fragen und den Austausch mit den Studierenden bleibt.
Wenn Sie kreative Ideen haben, wie Sie die Stunde interaktiver nutzen möchten, bringen Sie diese gerne mit ein!
Wir helfen gerne bei der Umsetzung.

Inhaltlich könnte Ihre Präsentation folgende Punkte abdecken:
\begin{itemize}
    \item Allgemeine Vorstellung des Unternehmens (Standort, Branche, Größe, Aufgabenfelder)
    \item Arbeitsfelder für MINT-Fachkräfte
    \item Einstiegsmöglichkeiten
    \item Falls vorhanden: Typischer Arbeitsalltag, Gehaltsstruktur
    \item Gerne: Persönlicher Werdegang, Einblick in Ihre persönliche (MINT-)Tätigkeit beim Unternehmen
\end{itemize} 

Diese Liste ist nicht vollständig und als Anregung gedacht.
Sie können Ihren Vortrag natürlich beliebig gestalten.

\subsection*{Katalog}

Zur PhysiKon wird ein digitaler Messekatalog erscheinen, der während und auch nach der PhysiKon-Woche kostenlos über unsere Homepage verfügbar ist.
Außerdem verlinken wir alle teilnehmenden Unternehmen auf unserer Homepage (https://physikon.pep-dortmund.org).
Für den Katalog und die Homepage brauchen wir (falls nicht schon eingereicht):
\begin{itemize}
    \item Logo des Unternehmens
    \item Adresse des Unternehmens
    \item Kontakt (Ansprechpartner, Email-Adresse) für Bewerbungen
    \item Branche
    \item Standorte
    \item Ungfähre Anzahl der Mitarbeiter
    \item Gewünschte Zusatzqualifikationen der Bewerber
    \item Kurzer Text (max. 500 Zeichen)
    \item Anzeigenbild (A5 hochkant)
\end{itemize}




\subsection*{Jobbörse}

Falls Sie aktuelle Stellenauschreibungen haben, veröffentlichen wir diese gerne in unserer PeP Jobbörse.
Die Jobbörse ist unabhängig von der PhysiKon immer über unsere Homepage erreichbar (https://pep-dortmund.org/jobboerse).

Senden Sie uns Ihre Stellenauschreibungen an jobboerse@pep-dortmund.org

\subsection*{Finanzielle Unterstützung}

Da wir in diesem Jahr mit dem Online-Format keine großen Kosten haben, gibt es keinen festen Ausstellerbeitrag.
Wir freuen uns trotzdem sehr, wenn Sie unseren gemeinnützigen Verein finanziell unterstützen!
Sie können für einen von Ihnen selbst gewählten Ausstellerbeitrag eine Rechnung oder eine Spendenquittung erhalten.

Unser Verein PeP et al. e.\,V. unterstützt die Fakultät Physik in Forschung und Lehre und fördert den Kontakt zwischen dem Fachbereich, seinen derzeitigen Studierenden und seinen Absolventen.
So organisieren wir zum Beispiel Workshops zu verschiedenen Themen, die jährliche Absolventenfeier der Fakultät und unsere Sommerakademie.
Außerdem finanzieren wir jährlich Stipendien für herausragende Studierende.
Mit Ihrem Beitrag unterstützen Sie unsere Projekte, die wir in ehrenamtlicher Arbeit für die Studierenden der Physik umsetzen.

Bei Fragen zu finanziellen Themen wenden Sie sich an lena.linhoff@tu-dortmund.de

\subsection*{Zeiteinteilung}

\begin{table}[h]
    \centering
\begin{tabular}{|c|l|}
    \hline
    \multicolumn{2}{|c|}{Montag, 19.04.2021} \\
    \hline
    \rowcolor{gray!10} 16 ­- 17 Uhr & Ritzenhoefer \& Company \\
    \rowcolor{gray!30} 17 - 18 Uhr & Bundesamt für Verfassungsschutz\\
    \hline
    \multicolumn{2}{|c|}{Dienstag, 20.04.2021} \\
    \hline
    \rowcolor{gray!10} 16 ­- 17 Uhr & Brunel \\
    \rowcolor{gray!30} 17 - 18 Uhr & Warth \& Klein Grant Thornton AG Wirtschaftsprüfungsgesellschaft\\
    \hline
    \multicolumn{2}{|c|}{Mittwoch, 21.04.2021} \\
    \hline
    \rowcolor{gray!10} 16 ­- 17 Uhr & Cohausz \& Florack \\
    \rowcolor{gray!30} 17 - 18 Uhr & Point8\\
    \hline
    \multicolumn{2}{|c|}{Donnerstag, 22.04.2021} \\
    \hline
    \rowcolor{gray!10}16 ­- 17 Uhr & EY Parthenon\\
    \rowcolor{gray!30}17 - 18 Uhr & thyssenkrupp Bilstein\\
    \hline
    \multicolumn{2}{|c|}{Freitag, 23.04.2021} \\
    \hline
    \rowcolor{gray!10}16 ­- 17 Uhr & Amprion \\
    \rowcolor{gray!30}17 - 18 Uhr & \\
    \hline
    \end{tabular}

\end{table}



Sollten noch Fragen offen sein, geben Sie uns gerne Bescheid (Email: physikon@pep-dortmund.org).
Wir freuen uns auf eine erfolgreiche PhysiKon 2021 mit Ihnen!

\vspace{1cm}
Ihr PhysiKon-Team und PeP et al. e.\,V.

\end{document}